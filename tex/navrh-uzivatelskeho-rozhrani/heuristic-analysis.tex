\section{Heuristická analýza}

\label{nur:test}

Testování uživatelského rozhraní je prováděno pomocí heuristické analýzy. Heuristická analýza obsahuje 10 bodů, které by mělo splňovat každé uživatelské rozhraní. Jak udává zdroj [7], jsou to tyto:

\begin{enumerate}
    \item Viditelnost stavu systému
        \begin{itemize}
            \item Systém nesmí zůstat zamrzlý a nereagovat na uživatelské vstupy.
            \item Zobrazit ukazatel průběhu.
            \item Uživatel musí být informován o~tom, co systém dělá.
        \end{itemize}
    \item Shoda mezi systémem a realitou
        \begin{itemize}
            \item Zachováni konvencí (a metafor) reálného světa.
            \item Ikony (a metafory) se musí chovat jako to, co zobrazují / na co odkazují.
            \item Do koše mohu věci nejen vyhodit, ale lze ho i vysypat, případně ho prohledat a vytáhnout už jednou vyhozenou věc.
            \item Pozor na překlady.
        \end{itemize}
    \item Minimální zodpovědnost (a stres)
        \begin{itemize}
            \item Nic se nemůže pokazit.
            \item Vždy je možnost vrátit se zpět do předchozího stavu.
            \item Uživatelé více experimentují a rychleji se učí.
            \item Uživatel ovládá systém, ne naopak.
            \item Zrušení dlouho trvajících operací.
            \item Potvrzování akcí.
            \item Varováni před provedením nevratné akce.
        \end{itemize}
    \item Shoda s~použitou platformou a obecnými standardy
        \begin{itemize}
            \item Program pod Windows vypadá a chová se jako pod Windows. Nijak jinak.
            \item Pokud to jde, použit standardní systémové komponenty.
            \item Systémové barvy a typy písem.
            \item Používat stejné termíny.
            \item Vysvětlit zkratky.
        \end{itemize}
    \item Prevence chyb
        \begin{itemize}
            \item Uživatel by neměl mít možnost zadat špatnou hodnotu.
            \item Následná chybová hláška to zachrání jen částečně.
            \item Pokud je způsob zadáváni z~principu složitější, je třeba to na místě vysvětlit.
            \item Zvýraznit povinné položky formulářů.
            \item Potvrzování akcí.
        \end{itemize}
    \item Kouknu a vidím
        \begin{itemize}
            \item Nezatěžovat uživatelovu paměť.
            \item Akce, které uživatel může momentálně provést by měly být viditelné a snadno dosažitelné.
            \item Stejně tak informace.
            \item Nepotřebné kontrolky a informace nepotřebujeme.
            \item Pozice uživatele.
            \item Pozice ve stromové struktuře.
        \end{itemize}
    \item Flexibilita a efektivita
        \begin{itemize}
            \item Zkušení vs. běžní uživatelé.
            \item To co zkušený uživatel pravidelně potřebuje, běžný kolikrát ani nepotřebuje vědět.
            \item Pokročilý mód.
            \item Klávesové zkratky / funkční klávesy.
            \item Makra.
            \item Klonování existujících záznamů.
            \item Jsou opravdu všechny akce/nastavení potřeba?
        \end{itemize}
    \item Minimalita (Klapky na očích)
        \begin{itemize}
            \item Zobrazovat pouze informaci, která aktuálně k~něčemu opravdu je.
            \item Čim méně možností uživatel má, tím rychleji koná.
            \item (Nepotřebná) grafika by neměla zastiňovat ovládání a účel.
            \item Cokoliv je zobrazeno soutěží o~uživatelovu pozornost.
            \item Je třeba nechat vyhrát to důležité.
        \end{itemize}
    \item Smysluplné chybové hlášky
        \begin{itemize}
            \item Nejlepší je nedojít do stavu kdy je třeba chybového hlášení.
            \item Chybové hlášení v~běžném jazyce – žádné kódy.
            \item Chybové hlášení by mělo popsat co se stalo špatně, jak se to stalo a jak tomu příště předejít.
            \item Případně možná řešení doporučit.
            \item Mělo by poučit (vzdělat) uživatele.
        \end{itemize}
    \item Pomoc a dokumentace
        \begin{itemize}
            \item Systém by měl být použitelný bez jakékoliv nápovědy, nicméně nápověda musí být.
            \item Musí podporovat funkci vyhledávání.
            \item Spíše než popisem by se měla zabývat příklady.
            \item Kontextová nápověda.
        \end{itemize}
\end{enumerate}

\subsection{Nalezené problémy}
Nalezené problémy jsou shrnuty v~tabulce \ref{tab:heuristic-analysis}. Každý z~nalezených problémů je hodnocen na stupnici 1 - 5, kde 1 znamená málo závažný problém a oproti tomu 5 značí velmi závažný problém. Pro všechny nalezené problémy je také navržen způsob řešení vedoucí k~odstranění problému. Všechny nalezené problémy byly před dalším vývojem aplikace vyřešeny.

\begin{table}[!h]
    \caption{Problémy nalezené při heuristické analýze}\label{tab:heuristic-analysis}
    \begin{tabulary}{1.0\textwidth}{|L|L|L|L|}
        \hline
        Bod analýzy & Nalezený problém & Závažnost problému (1 -- 5) & Řešení problému \\ \hline\hline
        4. & Sjednotit výrazy \textit{offer} a \textit{transaction}. & 2 & V~rámci celého webu se nyní používá výraz \textit{offer}. \\ \hline
        5. & Nejsou zvýrazněny povinné položky formulářů. & 3 & Byla přidána červená * k~povinným položkám formulářů. \\ \hline
        6. & Nelze rozpoznat, podle čeho se řadí. & 4 & Byl odstraněn odkaz z~aktuálního řazení. Došlo tak k~barevnému odlišení textů. \\ \hline
        10. & Chybí nápověda o~postupu práce s~nabídkou. & 4 & Byla přidána stránka \textit{Jak to funguje?} s~popisem a stavovým diagramem transakce. \\ \hline
    \end{tabulary}
\end{table}


