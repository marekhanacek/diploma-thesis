\section{REST API}

\subsection{Nabídky}
Níže jsou popsány všechny zdroje API, které si týkají nabídek.

\subsubsection{GET /api/offer}
Zdroj pro získávání nabídek. Je zde nastaveno výchozí stránkování po 10 záznamech.

\subsubsection{POST /api/offer}
Zdroj pro vytvoření nové nabídky.

\subsubsection{GET /api/offer/\{id\}}
Zdroj pro získání informací o jedné konkrétní nabídce.

\subsubsection{POST /api/offer/\{id\}/\{status\}}
Tento zdroj slouží pro změnu stavu nabídky. Parametr \texttt{status} lze nahradit jedním z těchto výrazů:
\begin{itemize}
    \item delete
    \item accept
    \item approve
    \item refuse
    \item already\_not\_interested
    \item offer\_again
    \item complete
\end{itemize}

\subsubsection{GET /api/offer/\{id\}/feedback}
Slouží k získávání dat o hodnoceních, které se týkají dané nabídky.

\subsubsection{POST /api/offer/\{id\}/feedback}
Slouží k přidání nového hodnocení.

\subsection{Měny}
Měny nelze pomocí API nijak upravovat, proto existují pouze 2 zdroje, pomocí nichž lze data získávat.
\subsubsection{GET /api/currency}
Výpis všech měn.
\subsubsection{GET /api/currency/\{id\}}
Výpis konkrétní měny dle zadaného parametru \texttt{id}.

\subsection{Uživatelé}
\subsubsection{GET /api/user}
Výpis všech uživatelů aplikace.
\subsubsection{GET /api/user/\{id\}}
Výpis konkrétního uživatele.
\subsubsection{GET /register-by-token/facebook/}
Tento zdroj slouží pro oznámení o registraci/přihlášení přes facebook pomocí mobilní aplikace, případně pomocí nějaké další třetí strany. Je potřeba uvést v url parametr \texttt{access-token}, pomocí kterého si aplikace zjistí další informace z API Facebooku.

\subsection{Jazyky}
Stejně jako měny, tak i jazyky nelze pomocí API nijak upravovat, proto existují pouze 2 zdroje, pomocí nichž lze data získávat.
\subsubsection{GET /api/language}
Výpis všech jazyků.
\subsubsection{GET /api/language/\{id\}}
Výpis konkrétního jazyka.
