\section{Přihlašování přes Facebook}

\begin{sloppypar}
Jedním z~požadavků na aplikaci bylo přihlášení pomocí sociální sítě Facebook. To Django přímo nepodporuje, ale existuje balíček Python Social Auth \cite{python-social-auth}, který poskytuje autentizační a autorizační mechanismy pro několik frameworků a několik služeb, přes které se lze přihlásit.

Implementace je následující:
\begin{enumerate}
    \item Nejprve je potřeba vytvořit aplikaci na stránce Facebook for Developers \cite{facebook-developers}, pomocí které bude probíhat přihlašování uživatelů. Zde se vygenerují dva klíče, které je potřeba později uvést v~konfiguraci balíčku -- \texttt{FACEBOOK\_KEY} a \texttt{FACEBOOK\_SECRET}.
    \item Poté je potřeba do konfiguračního souboru \texttt{settings.py} přidat několik důležitých konfiguračních parametrů. Viz ukázka kódu \ref{code:facebook-settings}.
    \item Následuje definice url adres: Do souboru \texttt{urls.py} je potřeba přidat tento řádek: \mbox{\texttt{url('', include('social\_django.urls', namespace='social'))}}, který zajistí propojení URL adres s~požadovanými pohledy.
    \item Poté je nutné spustit migrace databáze pomocí příkazu \\\texttt{python manage.py migrate}.
    \item Poté v~šabloně vytvoříme odkaz, pomocí kterého se budou uživatelé přihlašovat.
\end{enumerate}
\end{sloppypar}

\begin{listing}[h]
\caption{\label{code:facebook-settings}Konfigurace přihlášení přes Facebook}
\begin{minted}[frame=lines,bgcolor=codebg,fontsize=\footnotesize,linenos,breaklines]{Python}
AUTHENTICATION_BACKENDS = (
    'django.contrib.auth.backends.ModelBackend',
    'social_core.backends.facebook.FacebookOAuth2',
)

SOCIAL_AUTH_FACEBOOK_KEY = 'FACEBOOK_KEY'
SOCIAL_AUTH_FACEBOOK_SECRET = 'FACEBOOK_SECRET'
SOCIAL_AUTH_FACEBOOK_SCOPE = ['email']
SOCIAL_AUTH_FACEBOOK_PROFILE_EXTRA_PARAMS = {
    'locale': 'cs_CZ',
    'fields': 'id, name, email, age_range'
}

SOCIAL_AUTH_PIPELINE = (
    'social_core.pipeline.social_auth.social_details',
    'social_core.pipeline.social_auth.social_uid',
    'social_core.pipeline.social_auth.auth_allowed',
    'social_core.pipeline.social_auth.social_user',
    'social_core.pipeline.user.get_username',
    'social_core.pipeline.social_auth.associate_by_email',
    'social_core.pipeline.user.create_user',
    'web.pipeline.save_profile_picture',
    'web.pipeline.save_preferences_to_session',
    'social_core.pipeline.social_auth.associate_user',
    'social_core.pipeline.social_auth.load_extra_data',
    'social_core.pipeline.user.user_details'
)
\end{minted}
\end{listing}

\subsection{Autentizační proces}
Po návratu do aplikace z~API Facebooku aplikace postupně volá několik metod zvané roury\footnote{Roury (pipeline), jsou metody, které se volají postupně jedna za druhou, přičemž výstup jedné roury je vstupem druhé.}.
\begin{itemize}
    \item Aplikace definuje 2 roury:
    \begin{enumerate}
        \item \texttt{save\_profile\_picture} -- Metoda se stará o~uložení profilového obrázku.
        \item \texttt{save\_preferences\_to\_session} -- Nastavuje preference uživatelů do session.
    \end{enumerate}
\end{itemize}
