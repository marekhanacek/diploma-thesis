\chapter{Úvod do problematiky}
\label{problematics}

Na úvod je nutné definovat následující pojmy\footnote{V definicích lze domácí měnu chápat jako českou korunu, zahraniční měnu pak například Euro nebo americký dolar.}:
\begin{itemize}
    \item \textbf{Měnový kurz} -- Je cena jedné měny vyjádřená v~jednotkách měny jiné. Obvykle se udává jako poměr domácí měny ku měně zahraniční.
    \item \textbf{Nákupní kurz} -- Je kurz poskytovaný při \textbf{nákupu zahraniční měny}. Je zpravidla vyšší než prodejní kurz. Zákazník tak musí směnit větší množství peněz, což se stává nevýhodné.
    \item \textbf{Prodejní kurz} -- Je opakem kurzu nákupního. Prodejní kurz je kurz poskytovaný při \textbf{prodeji zahraniční měny}. Je zpravidla nižší než nákupní kurz. Zákazník tak dostane menší odbos peněž v~domácí měně.
    \item \textbf{Středový kurz} -- Je aritmetickým průměrem kurzu nákupního a prodejního.
\end{itemize}

Při směně peněz ve směnárnách je vždy zákazníkovy nabídnut ten méně výhodnější kurz. Zpětnou směnou peněz tak zákazník příjde o~část peněz\footnote{Není zde brána v~úvahu možná změna kurzu.}.

Kompromisem při osobní výměně peněz mezi lidmi je použití středového kurzu. Užitím středového kurzu tak dochází k~úspoře pěněz obou účastníků, jelikož existuje pouze jeden kurz, který je vždy výhodnější než ten, který zákazník dostane ve směnárně. Nemluvě o~provizním a manipulačním poplatku směnáren.

Tato jednoduchá úvaha je pak hlavní myšlenkou této aplikace.
