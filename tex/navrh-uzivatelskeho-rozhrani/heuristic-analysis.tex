\section{Testování uživatelského rozhraní}

\label{nur:test}

Testování uživatelského rozhraní budu provádět pomocí heuristické analýzy. Heuristická analýza obsahuje 10 bodů, které by mělo splňovat každé uživatelské rozhraní. Jsou to tyto \cite{nur-heu}

\begin{itemize}
    \item 1. Viditelnost stavu systému
        \begin{itemize}
            \item System nesmí zůstat zamrzlý a nereagovat na uživatelské vstupy
            \item Zobrazit progress bar
            \item Uživatel musí být informován o tom, co systém dělá
        \end{itemize}
    \item 2. Shoda mezi systémem a realitou
        \begin{itemize}
            \item Zachováni konvenci (a metafor) reálného světa
            \item Desktop...
            \item Ikony (a metafory) se musí chovat jako to, co zobrazují / na co odkazují
            \item Do koše mohu věci nejen vyhodit, ale lze ho i vysypat, případně ho prohledat a vytáhnout už jednou vyhozenou věc
            \item Pozor na překlady
        \end{itemize}
    \item 3. Minimální zodpovědnost (a stres)
        \begin{itemize}
            \item „Nic se nemůže pokazit“
            \item Vždy je možnost vrátit se zpět do předchozího stavu – Undo
            \item Nejlépe univerzální undo – historie
            \item Uživatele vice experimentuji a rychleji se učí
            \item Uživatel ovládá systém, ne naopak
            \item Zrušeni dlouho trvajících operaci (rollback)
            \item Potvrzováni akci
            \item Varováni před provedením nevratné akce
        \end{itemize}
    \item 4. Shoda s použitou platformou a obecnými standardy
        \begin{itemize}
            \item Program pod Windows vypadá a chová se jako pod Windows. Nijak jinak.
            \item Pokud to jde, použit standardní systémové komponenty
            \item Systémové barvy a fonty
            \item Používat stejné termíny
            \item Vysvětlit zkratky („Nastavte BPU!“)
        \end{itemize}
    \item 5. Prevence chyb
        \begin{itemize}
            \item Uživatel by neměl být možnost zadat špatnou hodnotu
            \item Následná chybová hláška to zachrání jen částečně
            \item Pokud je způsob zadáváni z principu složitější, je třeba to na místě vysvětlit (in-line help text, hint)
            \item Např. Sila hesla
            \item Zvýraznit povinné položky formulářů
            \item Potvrzováni akci
        \end{itemize}
    \item 6. Kouknu a vidím
        \begin{itemize}
            \item Nezatěžovat uživatelovu paměť
            \item Akce, které uživatel může momentálně provést by měli být viditelné a snadno dosažitelné
            \item Stejně tak informace
            \item Nepotřebné kontrolky a informace nepotřebujeme
            \item Pozice uživatele
            \item Krok ve wizardu (3 of 7)
            \item Pozice ve stromové struktuře (bread crumb)
        \end{itemize}
    \item 7. Flexibilita a efektivita
        \begin{itemize}
            \item Zkušení vs. běžní uživatele
            \item To co zkušený uživatel pravidelně potřebuje, běžný kolikrát ani nepotřebuje vědět
            \item Pokročilý (Advanced) mód
            \item Klávesové zkratky / funkční klávesy
            \item Makra
            \item Klonováni existujících záznamů (templates)
            \item Jsou opravdu všechny akce/nastaveni potřeba?
        \end{itemize}
    \item 8. Minimalita (Klapky na očích)
        \begin{itemize}
            \item Zobrazovat pouze informaci, která aktuálně k něčemu opravdu je.
            \item Čim méně možnosti uživatel má, tím rychleji koná
            \item (Nepotřebná) grafika by neměla zastiňovat ovládáni a účel
            \item Cokoliv je zobrazeno soutěži o uživatelovu pozornost
            \item Je třeba nechat vyhrát to důležité
        \end{itemize}
    \item 9. Smysluplné chybové hlášky
        \begin{itemize}
            \item Nejlepší je nedojít do stavu kdy je třeba chybového hlášení
            \item Chybové hlášení v běžném jazyce – žádné kódy
            \item Chybové hlášení by mělo popsat co se stalo špatně, jak se to stalo a jak tomu příště předejít.
            \item Případně možná řešení doporučit
            \item Mělo by poučit (vzdělat) uživatele
        \end{itemize}
    \item 10. Help a dokumentace
        \begin{itemize}
            \item Systém by měl být použitelný bez jakékoliv nápovědy, nicméně…
            \item Nápověda musí být
            \item Musí podporovat funkci vyhledáváni
            \item Spíše než popisem by se měla zabývat příklady
            \item Kontextová nápověda
        \end{itemize}
\end{itemize}

\subsection{Nalezené problémy}
Nalezené problémy jsou shrnuty v tabulce \ref{tab:heuristic-analysis}. Každý z nalezených problémů jsem hodnotil na stupnici 1 - 10, kde 1 znamená málo závažný problém a oproti tomu 10 značí velmi závažný problém. Pro všechny nalezené problémy také navrhuji způsob řešení, jak problém odstranit. Všechny nalezené problémy jsem před dalším vývojem aplikace vyřešil.

\begin{table}
    \caption{Problémy nalezené při heuristické analýze}\label{tab:heuristic-analysis}
    \begin{tabulary}{1.0\textwidth}{|L|L|L|L|}
        \hline
        Bod analýzy & Nalezený problém & Závažnost problému (1 - 10) & Řešení problému \\ \hline\hline
        4. & Sjednotit názvosloví offer a transaction & 10 & V rámci celého webu se nyní používá výraz offer \\ \hline
%        5. & Feedback - není ošetřen vstup & 10 & Přidat kontrolu, zda uživatel zadal komentář \\ \hline
        5. & Nejsou zvýrazněny povinné položky formulářů & 8 & Byla přidána červená * k povinným položkám formulářů. \\ \hline
%        6. & Zvážit přidání bread crumb & 3  & \\ \hline
        6. & Nelze rozpoznat, podle čeho se řadí & 8 & Odstraněn odkaz z aktuálního řazení. \\ \hline
        9. & Špatná chybová hláška na stránce Login required & 6 & Napsat, proč se uživatel musí přihlásit \\ \hline
        10. & Chybí nápověda o postupu práce s nabídkou & 7 & Stránka “How it works?” s krátkým popis + stavový diagram transakce \\ \hline
    \end{tabulary}
\end{table}


