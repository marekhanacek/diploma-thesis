\chapter{Úvod do problematiky}
\label{problematics}

Na úvod je nutné definovat následující pojmy\footnote{V definicích lze domácí měnu chápat jako českou korunu, zahraniční měnu pak například Euro nebo americký dolar.}:
\begin{itemize}
    \item \textbf{Měnový kurz} -- Je cena jedné měny vyjádřená v~jednotkách měny jiné. Obvykle je udávána jako poměr domácí ke zahraniční měně.
    \item \textbf{Nákupní kurz} -- Je kurz poskytovaný při \textbf{nákupu zahraniční měny}. Je zpravidla vyšší než prodejní kurz. Zákazník tak musí směnit větší množství peněz, což se stává nevýhodné.
    \item \textbf{Prodejní kurz} -- Je opakem kurzu nákupního. Prodejní kurz je kurz poskytovaný při \textbf{prodeji zahraniční měny}. Je zpravidla nižší než nákupní kurz. Zákazník tak dostane menší obnos peněz v~domácí měně.
    \item \textbf{Středový kurz} -- Je aritmetickým průměrem kurzu nákupního a prodejního.
\end{itemize}

Při směně peněz ve směnárnách je vždy zákazníkovi nabídnut ten méně výhodnější kurz. Zpětnou směnou tak zákazník přijde o~část peněz\footnote{Pokud nebereme v~úvahu možnou změnu kurzu.}.

Kompromisem je použití středového kurzu, který je mozné využít při osobní výměně peněz mezi lidmi. Užitím středového kurzu dochází k~úspoře peněz obou účastníků směny, jelikož existuje pouze jeden kurz, který je vždy výhodnější než ten, který zákazník dostane ve směnárně. Nemluvě o~provizi a manipulačním poplatku směnáren.

Tato jednoduchá úvaha je pak hlavní myšlenkou této práce. Cílem aplikace je propojit co možná nejvíce uživatelů, kteří hledají tu nejlepší nabídku na směnu a současně jsou ochotni směnit peníze osobně.
\\\\
V~současné chvíli je představa taková, že by aplikace měla poskytovat vyhledávání nabídek, výpis nabídek (textový výpis i mapu), detail konkrétní nabídky, detail uživatele, hodnocení uživatelů, stránku pro vytvoření nabídky a informační stránky webu.

Pro upřesnění této představy je v~následující kapitole provedena analýza podobných webových služeb. Analýza nemá za cíl jen upřesnit tyto požadavky, ale také dát určitý pohled na to, jak podobné problémy řeší jiné webové služby. Případně z~analýzy může vyplynout nová funkcionalita.