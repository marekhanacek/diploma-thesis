\section{REST API}
Pro implementaci REST API byl zvolen balíček Django REST framework \cite{django-rest-framework}. Základem tohoto balíčku jsou skupiny pohledů a serializery modelových tříd.

\subsection{Pohledy v~API}
\begin{sloppypar}
Pohledy v~kontextu API jsou realizovány pomocí tříd dědících od \texttt{rest\_framework.viewsets.GenericViewSet}. Ta, společně se serializery modelových tříd, poskytuje mechanismus pro vytvoření plně funkčního API pomocí opravdu malého počtu řádek kódu. Nejedná se o~jednotlivé pohledy, ale skupinu pohledů. Je nutné definovat proměnnou \texttt{queryset} určující o~jakou modelovou třídu se jedná a \texttt{serializer\_class} definující serializační třídu. Třída \texttt{rest\_framework.viewsets.GenericViewSet} dále nabízí podporu stránkování, řazení a filtrace.

V~aplikaci jsou pohledy dvojího typu:
Prvním typem jsou pohledy jen pro čtení. Takové pohledy mají jako předka třídu \texttt{rest\_framework.viewsets.ReadOnlyModelViewSet} a poskytují pouze zdroj pro výpis všech záznamů a zdroj pro výpis jednoho konkrétního záznamu. Konkrétně mezi takové pohledy patří:
\begin{itemize}
    \item \texttt{CurrencyViewSet} - Definuje zdroje týkající se měn.
    \item \texttt{LanguageViewSet} - Definuje zdroje týkající se jazyků.
    \item \texttt{UserViewSet} - Definuje zdroje týkající se uživatelů.
\end{itemize}

Druhým typem jsou pohledy určené pro čtení i zápis. Příkladem tohoto typu je \texttt{OfferViewSet}, jenž se stará o~množinu zdrojů týkajících se nabídek. Třída \texttt{OfferViewSet} oproti pohledům jen pro čtení navíc dědí od třídy \texttt{mixins.CreateModelMixin}, což umožňuje vytvářet nové nabídky.
\end{sloppypar}

\subsection{Serializery}
Serializery, jak název napovídá, slouží k~serializaci modelových tříd. Definují, která data se mají v~API zobrazovat. V~proměnné \texttt{fields} lze vyjmenovat všechny atributy modelových tříd, které mají být zobrazeny v~API. Dále pomocí proměnné \texttt{read\_only\_fields} lze definovat atributy, které jsou jen pro čtení a je tedy zamezeno tyto atributy přes API upravovat.
\\\\

\begin{sloppypar}
Kompletní dokumentace API je také dostupná na adrese \href{https://mhdip.herokuapp.com/api/docs/}{https://mhdip.herokuapp.com/api/docs/}.
\end{sloppypar}