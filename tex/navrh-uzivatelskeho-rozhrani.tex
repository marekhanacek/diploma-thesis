\chapter{Uživatelské rozhraní}
\label{nur}

Návrh uživatelského rozhraní vychází z~analýzy konkurenčních webů probíraných v~kapitole \ref{analyza} a z~požadavků na aplikaci z~kapitoly \ref{requirements}. Návrh se zaměřuje na tyto stránky aplikace:
\begin{itemize}
    \item Celkové rozvržení webu,
	\item Hlavní stránka,
	\item Detail nabídky,
	\item Můj profil -- profil aktuálně přihlášeného uživatele,
	\item Profil uživatele -- profil ostatních uživatelů,
	\item Nová nabídka,
	\item Informační stránky,
	\item Řazení výpisu nabídek,
	\item Změna adresy uživatele.
\end{itemize}
V~současné době stále stoupá procento přístupů na web z~mobilních zařízení, a proto je nutné se při návrhu uživatelského rozhraní soustředit nejen na verzi pro počítače s~vysokým rozlišením, ale také na verzi určenou pro mobilní zařízení.

\section{Celkové rozvržení webu}

\label{nur:layout}

Každá nabídka disponuje svou polohou a rádiusem, ve kterém je uživatel ochoten směnit své peníze. A pro takové použití je vhodné použít mapu. Mapa se však nehodí pro všechny podstránky aplikace. Proto jsem se rozhodl pro návrh 2 různých rozložení webu. Jeden s mapou a druhý bez mapy.

\subsection{Rozvržení webu s mapou}
Velmi povedené rozvržení stránky jsem vypozoroval na webu SReality.cz \ref{analyza:sreality}, čímž jsem se také inspiroval při tomto návrhu. Polovinu stránky tedy při tomto rozvržení zabírá mapa a druhou polovinu textová část. Toto rozvržení lze vidět na obrázku \ref{fig:tur:homepage}.

Textová část stránky obsahuje vedle hlavního obsahu hlavičku, kde se nachází logo a hlavní menu, a dále patičku s odkazy na informační stránky aplikace.

\subsection{Rozvržení webu bez mapy}
V podstatě se jedná o totožné rozvržení s tím rozdílem, že zde je mapa vynechána. Rozvržení lze vidět na obrázku \ref{fig:tur:offer-detail-no-map}.
\section{Hlavní stránka}

\label{nur:homepage}
Stránka používá rozvržení webu s~mapou. Podobu stránky lze vidět na obrázku \ref{fig:tur:homepage}.

\begin{figure}[!h]
    \centering
    \includegraphics[width=1.0\textwidth]{media/tur/homepage.png}
    \caption{Hlavní stránka aplikace}
    \label{fig:tur:homepage}
\end{figure}

\subsection{Mapa}
Mapa obsahuje ukazatel polohy uživatele, kruh znázorňující vyhledávací rádius a ukazatele poloh nabídek. Ukazatel polohy uživatele a poloh nabídek jsou barevně odlišeny.

\subsection{Textová část}
Textová část se skládá z~hlavičky, filtračního formuláře, seznamu nabídek s~možností seřazení, tlačítka pro vytvoření nové nabídky a patičky obsahující odkazy na informační stránky.

\subsection{Filtrace nabídek}
Formulář obsahuje tři základní parametry pomocí nichž lze filtrovat:
\begin{itemize}
	\item Měnu ze které chceme peníze směnit,
	\item měnu do které chceme peníze směnit,
	\item interval obnosu peněz.
\end{itemize}
Filtrace dále bere v~potaz i aktuální, respektive uživatelem zadanou polohu, a rádius. Filtrace se samozřejmě projevuje i v~mapě. Jsou tedy zobrazeny jen ty nabídky, které odpovídají zadaným filtrům.

\subsection{Řazení nabídek}
Vidno na obrázku \ref{fig:tur:sorting}. Nabídky lze seřadit čtyřmi různými způsoby:
\begin{itemize}
	\item Vzestupně dle vzdálenosti uživatelovy polohy od centra nabídky.
	\item Vzestupně dle obnosu peněz nabídek.
	\item Sestupně dle hodnocení uživatelů.
	\item Dle kombinace více kritérií nabídek. Tento způsob blíže popisuji v~následující podkapitole.
\end{itemize}

\begin{figure}[!h]
    \centering
    \includegraphics[width=1.0\textwidth]{media/tur/sorting.png}
    \caption{Výpis nabídek s~možností seřazení}
    \label{fig:tur:sorting}
\end{figure}

\subsubsection{Řazení nabídek dle kombinace více kritérií}
Hodnocení se skládá ze dvou složek:
\begin{itemize}
	\item Hodnocení uživatele:
        \begin{itemize}
            \item Jestliže má uživatel měně než dvě hodnocení, pak je tato hodnota rovna 0.
            \item Jestliže má uživatel dvě a více hodnocení, pak se hodnota odvíjí od průměrného počtu hvězdiček. Za špatné hodnocení je penalizován, za dobré hodnocení pak odměněn kladnými body. Konkrétně\footnote{Uvedené hodnocení a penalizace jsou založeny na základě experimentálního zkoumání řazení s~velkým množstvím testovacích dat.}:
            \begin{itemize}
                \item 1 hvězdička $\rightarrow$ penalizace -1.5,
                \item 2 hvězdičky $\rightarrow$ penalizace -1,
                \item 3 hvězdičky $\rightarrow$ hodnocení je rovno 0.6,
                \item 4 hvězdičky $\rightarrow$ hodnocení je rovno 0.8,
                \item 5 hvězdiček $\rightarrow$ hodnocení je rovno 1.
            \end{itemize}
        \end{itemize}
	\item Vzdálenost od uživatele:
        \begin{itemize}
            \item Za vzdálenost od uživatele může nabídka dostat 0 až 1 bod.
            \item Hodnota se počítá na základě vzdálenosti nejbližší a nejvzdálenější nabídky a dále na základě vzdálenosti právě hodnocené nabídky.
            \item TODO. Kód pro výpočet této hodnoty lze vidět na ukázce kódu \ref{code:sorting-distance}.
            \item Čím blíže je tedy nabídka k~zákazníkovi tím lepší hodnocení obdrží.
            \item \textit{Příklad}: V~případě tři nabídek ve vzdálenostech 15~km, 2~km a 28~km, budou hodnocení 0.5, 1 a 0.
        \end{itemize}
\end{itemize}

\begin{listing}[htbp]
\caption{\label{code:sorting-distance}Funkce pro výpočet hodnocení vzdálenosti od uživatele}
\begin{minted}[frame=lines,bgcolor=codebg,fontsize=\footnotesize,linenos,breaklines]{Python}
def get_distance_rating(offer, min_distance, max_distance, lat, lng):
    """
    Keyword arguments:
    offer -- concrete offer
    min_distance -- distance of nearest offer
    max_distance -- distance of the furthest offer
    lat -- input latitude
    lng -- input longitude
    """
    distance_from_min = get_offer_distance_from(offer, lat, lng) - min_distance
    min_max_distance = max_distance - min_distance
    percent = min_max_distance / 100
    return 1 - ((distance_from_min / percent) / 100)
\end{minted}
\end{listing}

\subsection{Mobilní verze}
Stejně jako při návrhu desktopové verze, tak i při návrhu mobilní verze je návrh inspirován webem Sreality.cz, který byl analyzován v~kapitole \ref{analyza:sreality}. Zobrazuje se tedy vždy pouze mapa nebo textová část. Viz obrázek \ref{fig:tur:homepage-mobile} na straně \pageref{fig:tur:homepage-mobile}.

\begin{figure}[!h]
    \centering
    \includegraphics[width=1.0\textwidth]{media/tur/homepage-mobile.png}
    \caption{Mobilní verze aplikace}
    \label{fig:tur:homepage-mobile}
\end{figure}

\section{Detail nabídky}

\label{nur:detail}

Existují 2 verze zobrazení detailu nabídky. A to zobrazení spolu s mapou a zobrazení na samostatné stránce.

\subsection{Zobrazení spolu s mapou}
Tímto způsobem se zobrazují pouze nabídky, které jsou volné. V mapě se zobrazuje vzdálenost mezi polohou uživatele a polohou nabídky. V textové části pak základní informace o nabídce a dále také informace o uživateli. Vidíme zde nejen kdo nabídku vytvořil, ale také jeho hodnocení. Stránka dále obsahuje tlačítko, které slouží k projevení zájmu o danou nabídku. V případě, že již uživatel má nějaké zkušenosti s tímto uživatelem, jsou zobrazeny nabídky, které mají společné. Z detailu nabídky se lze vrátit zpět na výpis nabídek pomocí tlačítka \textit{Back}. Detail nabídky lze vidět na obrázku \ref{fig:tur:offer-detail-map}.

\begin{figure}[h]
    \centering
    \includegraphics[width=1.0\textwidth]{media/tur/offer-detail-map.png}
    \caption{Detail nabídky s mapou}
    \label{fig:tur:offer-detail-map}
\end{figure}

\subsection{Zobrazení na samostatné stránce}
Toto zobrazení obsahuje totožné informace jako zobrazení s mapou. Při rozvržení bez mapy však máme více prostoru a detail nabídky je tedy uspořádán jinak. Na toto zobrazení se uživatel dostane ze svého uživatelského profilu. Detail nabídky s tímto rozvržením lze vidět na obrázku \ref{fig:tur:offer-detail-no-map}

\begin{figure}[h]
    \centering
    \includegraphics[width=1.0\textwidth]{media/tur/offer-detail-no-map.png}
    \caption{Detail nabídky bez mapy}
    \label{fig:tur:offer-detail-no-map}
\end{figure}
\section{Můj profil}
\label{nur:my-profile}

Stránka \textit{Můj profil} obsahuje základní informace o~uživateli, jako je jeho aktuální adresa, poloměr a hodnocení uživatele.
Profil uživatele zobrazuje všechny aktuální nabídky, které jsou dále rozděleny na sekce dle toho, zda daná nabídka čeká na akci právě přihlášeného uživatele nebo na akci někoho jiného. Další sekcí jsou již dokončené nabídky s~přehledem hodnocení, případně s~odkazem na přidání hodnocení. Podoba stránky je na obrázku \ref{fig:tur:my-profile} na straně \pageref{fig:tur:my-profile}.

\begin{figure}[!h]
    \centering
    \includegraphics[width=1.0\textwidth]{media/tur/my-profile.png}
    \caption{Můj profil}
    \label{fig:tur:my-profile}
\end{figure}
\section{Profil uživatele}

\label{nur:user-profile}

Obsahuje základní informace o uživateli, jeho hodnocení a nabídky které má aktuálně přihlášený uživatel s daným uživatelem společné. Rozvržení je velmi podobné podstránce \textit{Můj profil}. Podoba stránky je na obrázku \ref{fig:tur:user-profile}.

\begin{figure}[h]
    \centering
    \includegraphics[width=1.0\textwidth]{media/tur/user-profile.png}
    \caption{Profil uživatele}
    \label{fig:tur:user-profile}
\end{figure}
\section{Nová nabídka}

\label{nur:new-offer}

Obrázek \ref{fig:tur:new-offer}. Pokud uživatel nenajde nabídku, které mu vyhovuje, může použít právě tuto stránku pomocí které vytvoří novou nabídku. Formulář pro přidání nové nabídky obsahuje:
\begin{itemize}
    \item Měny, které budou směněny
    \item Částky v jednotlivých měnách
	\item Adresu pro poskytnutí nabídky
	\item Rádius
	\item Komentář k nabídce
	\item Mapa - Pro lepší orientaci při zadávání adresy a rádiusu je v pravé části stránky umístěna mapa, která zobrazuje aktuální výběr adresy a rádiusu.
\end{itemize}

\begin{figure}[h]
    \centering
    \includegraphics[width=1.0\textwidth]{media/tur/new-offer.png}
    \caption{Můj profil}
    \label{fig:tur:new-offer}
\end{figure}
\section{Informační stránky}
\label{nur:other-sections}

Všechny informační stránky mají stejné rozvržení, které je vidět na obrázku \ref{fig:tur:contact} na straně \pageref{fig:tur:contact}.

Mezi tyto stránky patří:
\begin{itemize}
    \item \textbf{Podmínky užití} -- Informační stránka s~podmínkami užití.
%    \item \textbf{Jak to funguje?} -- Obsahuje krátký popis toho, k čemu stránka slouží a jak ji používat.
    \item \textbf{Kontakty} -- Obsahuje kontaktní informace pro případně připomínky uživatelů.
\end{itemize}

\begin{figure}[!h]
    \centering
    \includegraphics[width=1.0\textwidth]{media/tur/contact.png}
    \caption{Informační stránka Kontakty}
    \label{fig:tur:contact}
\end{figure}
%\section{Hodnocení uživatelů}
\label{nur:feedback}

TODO
\section{Změna adresy}

\label{nur:address-change}

Změnu adresy lze provést dvěma způsoby: přes uživatelský profil nebo přímo tlačítkem \textbf{Change your address}. Po použití tlačítka \textbf{Change your address} se zobrazí vyskakovací okno \ref{fig:tur:address-change}, kde si uživatel zadá svou polohu (zadáním adresy nebo pomocí tlačítka moje poloha) a zvolí rádius pro vyhledávání. Poté pomocí tlačítka \textbf{Save} adresu uloží.

\begin{figure}[h]
    \centering
    \includegraphics[width=1.0\textwidth]{media/tur/address-change.png}
    \caption{Změna adresy}
    \label{fig:tur:address-change}
\end{figure}
\section{Stavy nabídky}
\label{nur:status}

Nabídka může být v~jednom z~pěti stavů:
\begin{itemize}
    \item \textbf{Čeká na druhého uživatele} -- Nabídka byla vytvořena a čeká na druhého účastníka směny.
    \item \textbf{Čeká na schválení} -- Nabídka již má oba účastníky směny a momentálně čeká na schválení uživatele uživatelem, který nabídku vytvořil.
    \item \textbf{Připravena ke směně} -- V~tomto momentě jsou zobrazeny kontaktní údaje a uživatelé se domlouvají na směně peněz.
    \item \textbf{Dokončená} -- Nabídka je kompletní.
    \item \textbf{Smazaná} -- Nabídka byla smazána uživatelem, který ji vytvořil.
\end{itemize}
Jak lze přecházet ze stavu do stavu znázorňuje stavový diagram \ref{fig:implementation:state-diagram}.
\begin{figure}[h]
    \centering
    \includegraphics[width=1.0\textwidth]{media/state-diagram-cs}
    \caption{Stavový diagram nabídky}
    \label{fig:implementation:state-diagram}
\end{figure}

Se změnami stavů objednávky úzce souvisí odesílání e-mailů uživatelům, což popisuje následující pod kapitola.
\section{Odesílání emailů}

\label{nur:mail}

Aplikace odesílá emaily při každé změně stavu nabídky. Konkrétně jsou to tyto:
\begin{itemize}
    \item \textbf{Akceptování nabídky} -- Odesílá se autorovi nabídky v momentě akceptování nabídky jiným uživatelem.
    \item \textbf{Uživatel již nemá zájem o nabídku} -- Odesílá se autorovi nabídky v momentě, kdy druhý uživatel již nemá zájem o nabídku.
    \item \textbf{Akceptování uživatele} -- Odesílá se uživateli, který se k nabídce připojil v momentě akceptování autorem nabídky.
    \item \textbf{Odmítnutí uživatele} -- Odesílá se uživateli, který se k nabídce připojil v momentě odmítnutí daného uživatele autorem nabídky.
    \item \textbf{Již není připravena ke směně} -- Tento email mohou obdržet oba uživatelé a to v momentě, kdy je nabídka ve stavu \textit{Připravena ke směně}, ale jeden z nich nabídku odmítne.
    \item \textbf{Nabídka byla dokončena} -- Odesílá se v momentě označení objednávky jako \textit{Dokončené} jedním z uživatelů.
    \item \textbf{K nabídce bylo přidáno hodnocení} -- Odesílá se v momentě přidání hodnocení jedním z uživatelů.
\end{itemize}
\section{Heuristická analýza}

\label{nur:test}

Testování uživatelského rozhraní je prováděno pomocí heuristické analýzy. Heuristická analýza obsahuje 10 bodů, které by mělo splňovat každé uživatelské rozhraní. Jsou to tyto \cite{nur-heu}:

\begin{itemize}
    \item 1. Viditelnost stavu systému
        \begin{itemize}
            \item Systém nesmí zůstat zamrzlý a nereagovat na uživatelské vstupy
            \item Zobrazit progress bar
            \item Uživatel musí být informován o~tom, co systém dělá
        \end{itemize}
    \item 2. Shoda mezi systémem a realitou
        \begin{itemize}
            \item Zachováni konvencí (a metafor) reálného světa
            \item Desktop...
            \item Ikony (a metafory) se musí chovat jako to, co zobrazují / na co odkazují
            \item Do koše mohu věci nejen vyhodit, ale lze ho i vysypat, případně ho prohledat a vytáhnout už jednou vyhozenou věc
            \item Pozor na překlady
        \end{itemize}
    \item 3. Minimální zodpovědnost (a stres)
        \begin{itemize}
            \item „Nic se nemůže pokazit“
            \item Vždy je možnost vrátit se zpět do předchozího stavu – Undo
            \item Nejlépe univerzální undo – historie
            \item Uživatelé více experimentují a rychleji se učí
            \item Uživatel ovládá systém, ne naopak
            \item Zrušení dlouho trvajících operací (rollback)
            \item Potvrzování akcí
            \item Varováni před provedením nevratné akce
        \end{itemize}
    \item 4. Shoda s~použitou platformou a obecnými standardy
        \begin{itemize}
            \item Program pod Windows vypadá a chová se jako pod Windows. Nijak jinak.
            \item Pokud to jde, použit standardní systémové komponenty
            \item Systémové barvy a fonty
            \item Používat stejné termíny
            \item Vysvětlit zkratky („Nastavte BPU!“)
        \end{itemize}
    \item 5. Prevence chyb
        \begin{itemize}
            \item Uživatel by neměl mít možnost zadat špatnou hodnotu
            \item Následná chybová hláška to zachrání jen částečně
            \item Pokud je způsob zadáváni z~principu složitější, je třeba to na místě vysvětlit (in-line help text, hint)
            \item Např. Síla hesla
            \item Zvýraznit povinné položky formulářů
            \item Potvrzování akcí
        \end{itemize}
    \item 6. Kouknu a vidím
        \begin{itemize}
            \item Nezatěžovat uživatelovu paměť
            \item Akce, které uživatel může momentálně provést by měli být viditelné a snadno dosažitelné
            \item Stejně tak informace
            \item Nepotřebné kontrolky a informace nepotřebujeme
            \item Pozice uživatele
            \item Krok ve wizardu (3 of 7)
            \item Pozice ve stromové struktuře (bread crumb)
        \end{itemize}
    \item 7. Flexibilita a efektivita
        \begin{itemize}
            \item Zkušení vs. běžní uživatelé
            \item To co zkušený uživatel pravidelně potřebuje, běžný kolikrát ani nepotřebuje vědět
            \item Pokročilý (Advanced) mód
            \item Klávesové zkratky / funkční klávesy
            \item Makra
            \item Klonování existujících záznamů (templates)
            \item Jsou opravdu všechny akce/nastavení potřeba?
        \end{itemize}
    \item 8. Minimalita (Klapky na očích)
        \begin{itemize}
            \item Zobrazovat pouze informaci, která aktuálně k~něčemu opravdu je.
            \item Čim méně možností uživatel má, tím rychleji koná
            \item (Nepotřebná) grafika by neměla zastiňovat ovládáni a účel
            \item Cokoliv je zobrazeno soutěží o~uživatelovu pozornost
            \item Je třeba nechat vyhrát to důležité
        \end{itemize}
    \item 9. Smysluplné chybové hlášky
        \begin{itemize}
            \item Nejlepší je nedojít do stavu kdy je třeba chybového hlášení
            \item Chybové hlášení v~běžném jazyce – žádné kódy
            \item Chybové hlášení by mělo popsat co se stalo špatně, jak se to stalo a jak tomu příště předejít.
            \item Případně možná řešení doporučit
            \item Mělo by poučit (vzdělat) uživatele
        \end{itemize}
    \item 10. Help a dokumentace
        \begin{itemize}
            \item Systém by měl být použitelný bez jakékoliv nápovědy, nicméně…
            \item Nápověda musí být
            \item Musí podporovat funkci vyhledávání
            \item Spíše než popisem by se měla zabývat příklady
            \item Kontextová nápověda
        \end{itemize}
\end{itemize}

\subsection{Nalezené problémy}
Nalezené problémy jsou shrnuty v~tabulce \ref{tab:heuristic-analysis}. Každý z~nalezených problémů je hodnocen na stupnici 1 - 5, kde 1 znamená málo závažný problém a oproti tomu 5 značí velmi závažný problém. Pro všechny nalezené problémy je také navržen způsob řešení vedoucí k~odstranění problému. Všechny nalezené problémy byly před dalším vývojem aplikace vyřešeny.

\begin{table}[!h]
    \caption{Problémy nalezené při heuristické analýze}\label{tab:heuristic-analysis}
    \begin{tabulary}{1.0\textwidth}{|L|L|L|L|}
        \hline
        Bod analýzy & Nalezený problém & Závažnost problému (1 -- 5) & Řešení problému \\ \hline\hline
        4. & Sjednotit výrazy \textit{offer} a \textit{transaction}. & 2 & V~rámci celého webu se nyní používá výraz \textit{offer}. \\ \hline
        5. & Nejsou zvýrazněny povinné položky formulářů. & 3 & Byla přidána červená * k~povinným položkám formulářů. \\ \hline
        6. & Nelze rozpoznat, podle čeho se řadí. & 4 & Byl odstraněn odkaz z~aktuálního řazení. Došlo tak k~barevnému odlišení textů. \\ \hline
        10. & Chybí nápověda o~postupu práce s~nabídkou. & 4 & Byla přidána stránka \textit{Jak to funguje?} s~popisem a stavovým diagramem transakce. \\ \hline
    \end{tabulary}
\end{table}



\section{Testování uživatelského rozhraní s~uživateli}
\label{user-testing}
Pro testování s~uživateli bylo připraveno 9 úkolů, u~kterých se očekává, že budou ty nejčastější úkony na stránce. Testování bylo provedeno s~pěti osobami.
\subsection{Úkoly}

\begin{enumerate}
    \item Přihlaste se do systému.
    \item Změňte jazyk na češtinu.
    \item Zobrazte kontaktní informace.
    \item Zobrazte profil uživatele Jim Beam.
    \item Změňte adresu na vaši aktuální.
    \item Máte k~dispozici 10 000 českých korun a chcete je směnit na eura. Nalezněte nejlepší nabídku a potvrďte zájem o~tuto nabídku.
    \item Žádná z~nabídek se vám nelibí, vytvořte ji.
    \item Představte si situaci, že jste vyměnili s~jiným uživatelem peníze. Nabídka je tedy ve stavu \textit{Připravena ke směně (Ready to exchange)}. Zaznamenejte tuto aktivitu do systému.
    \item Přidejte hodnocení ke směně v~předchozím kroku. Nabídka je nyní ve stavu \textit{Dokončená (Finished)}.
\end{enumerate}

\subsection{Představení testovaných osob}

\begin{itemize}
    \item Osoba A~-- 22 let, pracující, směnárnu použila jednou v~životě, zkušený uživatel v~oblasti webu a sociálních sítí.
    \item Osoba B -- 53 let, pracující, méně zkušený uživatel v~oblasti webu a sociálních sítí, směnárnu užívá 1x za rok.
    \item Osoba C -- 49 let, pracující, méně zkušený uživatel v~oblasti webu a sociálních sítí, směnárnu nikdy nepoužila.
    \item Osoba D -- 23 let, student FEL ČVUT, směnárnu používá několikrát za rok, zkušený uživatel v~oblasti webu, hodně cestuje.
    \item Osoba E -- 30 let, pracující, směnárnu používá několikrát do roka, hodně cestuje, zkušený uživatel v~oblasti webu.
\end{itemize}

\section{Shrnutí testování}
Výsledky testování a nalezené problémy jsou uvedeny v~tabulce \ref{tab:test-results}. Z~testování vyplynuly tyto změny v~aplikaci, které byly do aplikace zakomponovány:

\begin{itemize}
%    \item Zvážit přidání vyhledávacího pole do hlavičky.
    \item Tlačítko pro potvrzení akce -- přidat ikonu potvrzení.
    \item Přidat tlačítko pro změnu adresy i do podstránek.
    \item V~detail nabídky byly provedeny tyto změny:
    \begin{itemize}
        \item Zvýraznit pole s~informacemi o~nabídce.
        \item Jiné barvy pro různá tlačítka.
    \end{itemize}
\end{itemize}

\begin{table}[h]
    \caption{Problémy nalezené při testování s~uživateli}\label{tab:test-results}
    \begin{tabulary}{1.0\textwidth}{|c|c|L|}
        \hline
        \textbf{Osoba} & \textbf{Úkol} & \textbf{Popis problému} \\ \hline\hline
        A~& 5 & Uživatel jako jediný z~testovaných měnil adresu přes stránku \textit{Změna údajů (Change preferences)} v~uživatelském profilu. \\ \hline
%        A & 9 & Dle uživatele je detail nabídky nepřehledný.  \\ \hline

        B & 4 & Uživatel hledal vyhledávací pole, kam může zadat hledaný výraz. Později si všiml, že se na úvodní stránce nachází odkaz na Jimův profil. \\ \hline
        B & 6 & Při vyhledávání nabídek si uživatel myslel, že tlačítko New offer slouží pro potvrzení vyhledávání. \\ \hline
        B & 8 & Uživatel velmi dlouho hledal tlačítko pro potvrzení transakce. Hodně pozornosti věnoval sekci \textit{Historie s~uživatelem (History with user)}. \\ \hline

        C & 2 & Uživatel klikl na ikonu \textit{Google Translator} v~prohlížeči pro přeložení.  \\ \hline
        C & 3 & Úkol trval nad očekávání velmi dlouho.  \\ \hline
        C & 5 & Uživatel nevěděl, kde má změnit svou adresu. Později tlačítko našel.  \\ \hline
        C & 6 & Vyhledání nejlepší nabídky proběhlo v~pořádku ale, nemohla najít tlačítko pro potvrzení nabídek, což bylo způsobeno neznalostí anglického jazyka. \\ \hline

        E & 5 & Možnost změny adresy hledal na stránce z~předchozího úkolu -- \textit{Profil uživatele}.  \\ \hline
        E & 6 & Poukázal na to, že by v~tlačítku pro potvrzení měla být ikona.  \\ \hline
    \end{tabulary}
\end{table}

