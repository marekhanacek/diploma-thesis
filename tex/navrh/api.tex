\pagebreak
\section{Návrh REST API}\label{rest-api-design}
V~následujících podkapitolách jsou uvedeny všechny zdroje spolu s~příslušnou HTTP metodou. Navržené API se drží principů REST API.

\subsection{GET /api/offer}
Zdroj pro získávání nabídek s~možností filtrace, stránkování a řazení pomocí parametrů v~URL adrese. Těmito parametry jsou:
\begin{itemize}
    \item Povinné parametry:
    \begin{itemize}
        \item \texttt{currency\_from} -- Měna z,
        \item \texttt{currency\_to} -- Měna do,
        \item \texttt{lat} -- Zeměpisná šířka,
        \item \texttt{lng} -- Zeměpisná délka,
        \item \texttt{radius} -- Rádius pro vyhledávání.
    \end{itemize}
    \item Nepovinné parametry:
    \begin{itemize}
        \item \texttt{amount\_from} -- Rozsah peněz od,
        \item \texttt{amount\_to} -- Rozsah peněz do,
        \item \texttt{page} -- Stránka.
    \end{itemize}
\end{itemize}

\subsection{POST /api/offer}
Zdroj pro vytvoření nové nabídky. V~těle požadavku je potřeba uvést tyto parametry:
\begin{itemize}
    \item \texttt{lat} -- Zeměpisná šířka,
    \item \texttt{lng} -- Zeměpisná délka,
    \item \texttt{radius} -- Rádius pro vyhledávání,
    \item \texttt{amount} -- Rozsah peněz od,
    \item \texttt{comment} -- Komentář k~nabídce,
    \item \texttt{currency\_from} -- Měna z,
    \item \texttt{currency\_to} -- Měna do,
    \item \texttt{address} -- Adresa.
\end{itemize}

\subsection{GET /api/offer/\{id\}}
Zdroj pro získání informací o~jedné konkrétní nabídce.

\subsection{POST /api/offer/\{id\}/\{status\}}
Tento zdroj slouží pro změnu stavu nabídky. Parametr \texttt{status} lze nahradit jedním z~těchto výrazů:
\begin{itemize}
    \item delete,
    \item accept,
    \item approve,
    \item refuse,
    \item already\_not\_interested,
    \item offer\_again,
    \item complete.
\end{itemize}

\subsection{GET /api/offer/\{id\}/feedback}
Slouží k~získávání dat o~hodnoceních, které se týkají dané nabídky. Výstup nelze řadit, filtrovat ani stránkovat.

\subsection{POST /api/offer/\{id\}/feedback}
Slouží k~přidání nového hodnocení. Vstupní parametry lze vypozorovat z~definice modelové třídy Feedback.

\subsection{GET /api/currency}
Výpis všech měn. Pokud je počet záznamů větší než 10, pak je výpis stránkován po 10 záznamech.

\subsection{GET /api/currency/\{id\}}
Výpis konkrétní měny dle zadaného parametru \texttt{id}.

\subsection{GET /api/user}
Výpis všech uživatelů aplikace. Výpis je stránkován po 10 záznamech. Stránkování lze měnit pomocí parametru \texttt{page}.

\subsection{GET /api/user/\{id\}}
Výpis konkrétního uživatele dle zadaného parametru \texttt{id}.

\subsection{GET /api/user/\{id\}/finished\_offers}
Seznam dokončených nabídek daného uživatele.

\subsection{GET /api/user/\{id\}/user\_reaction}
Seznam nabídek čekající na reakci daného uživatele.

\subsection{GET /api/user/\{id\}/other\_user\_reaction}
Seznam nabídek čekajících na reakci jiných uživatelů.

\subsection{GET /api/user/login/facebook}
Tento zdroj slouží pro oznámení o~registraci/přihlášení přes Facebook pomocí mobilní aplikace. Je potřeba uvést v~url parametr \texttt{access-token}, pomocí kterého si aplikace zjistí další potřebné informace z~API Facebooku.

\subsection{GET /api/language}
Výpis všech jazyků. Pokud je počet záznamů větší než 10, pak je výpis stránkován po 10 záznamech.

\subsection{GET /api/language/\{id\}}
Výpis konkrétního jazyka dle zadaného parametru \texttt{id}.
