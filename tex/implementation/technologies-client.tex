\section{Technologie na klientské straně}

\subsection{HTML}
Značkovací jazyk HTML je používaný pro tvorbu webových stránek, které jsou propojeny hypertextovými odkazy. V mé aplikaci používám HTML ve verzi 5. HTML5 obsahuje oproti předchozím verzím nové HTML značky, které sémanticky definují strukturu stránky. Další výhodou je přímá podpora přehrávání multimédií v prohlížeči.

\subsection{CSS}
Stejně jako HTML je i jazyk CSS nedílnou součástí každé webové aplikace. CSS slouží pro popis způsobu zobrazení elementů napsaných v HTML.

\subsubsection{Bootstrap}
V aplikaci používám CSS framework Bootstrap, který přináší spoustu předdefinovaných kaskádových stylů pro tvorbu webových aplikací. V mé aplikaci používám Bootstrap ve verzi 3.3.7. Verze 4 je v době, kdy píši tuto diplomovou práci prozatím v aplha verzi, a proto jsem se použití této verze raději vyhl. Z komponent, které framework Bootstrap nabízí v mé aplikaci využívám tyto:
\begin{itemize}
    \item Glyphicons - Ikony
	\item Buttons - Tlačítka různých stylů
	\item Navbar - Navigace v horní části stránky s podporou responzivního designu
	\item Popup - Vyskakovací okno, které používám například pro změnu adresy uživatele
	\item Tooltip - Informační okénko, které se zobrazí při najetí myší na nějaký element stránky
	\item Tab - Lišty
\end{itemize}

\subsubsection{LESS}
LESS je preprocessor pro CSS. Obohacuje CSS o spoustu další funkčnosti:

\begin{itemize}
    \item Práce s proměnnými - Vhodné pro znovupoužití. Například pro ukládání často používaných barev.
	\item Vnořování CSS pravidel do sebe
	\item Dědění pravidel - Lze dědit pravidla od jiného CSS stylu
	\item Mixins - Obdoba dědění pravidel - umožňují vložit vlastnosti třídy do jiné
	\item Výpočty
	\item Operace s barvami - Lze vytvářet jiné odstíny barev
\end{itemize}

\subsection{JavaScript}

\subsubsection{jQuery}
jQuery je knihovna obohacující JavaScript o několik podpůrných funkcí, které programátorům usnadňují vývoj na klientské straně.

\subsubsection{Google mapy}
V mé aplikaci je potřeba zobrazovat nabídky na mapě, na což se perfektně hodí mapy od společnosti Google. V aplikaci používám zejména zobrazení nabídek v mapě s možností otevření informačního okna s detailem nabídky, převod z adresy na souřadnice a zpět, zobrazování kruhů v mapě a další.