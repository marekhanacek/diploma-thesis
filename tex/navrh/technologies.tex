\section{Technologie}
Použité technologie lze rozdělit do tří částí:
\begin{itemize}
    \item databázové technologie,
    \item technologie na klientské straně,
    \item technologie na serverové straně.
\end{itemize}

\subsection{Databázové technologie}
Aplikace používá pro ukládání dat databázový server \textbf{MySQL}. Vzhledem k~použití zvoleného frameworku Django je možné databázový server velmi jednoduše zaměnit například za \textbf{PostgreSQL}, \textbf{Oracle database} nebo \textbf{SQLite}.

\subsection{Technologie na klientské straně}
Vedle standardních technologií používaných při tvorbě webových aplikací, jako jsou \textbf{HTML}, \textbf{CSS} a \textbf{JavaScript}, aplikace používá \textbf{CSS framework Bootstrap}, který obsahuje velké množství předdefinovaných nejen kaskádových stylů, ale také JavaSriptu. Pro CSS je dále použit \textbf{preprocessor LESS}, který přináší přehlednější CSS kód, který je obohacen o~řadu vylepšení. Pro dynamické efekty na stránce a příjemnější použití JavaScriptu aplikace používá knihovnu \textbf{jQuery}. S~JavaScriptem je dále spojena knihovna podporující použití \textbf{Google map} na webu.

\subsubsection{Google maps}
V~aplikaci je potřeba zobrazovat nabídky na mapě. K tomu bylo vybráno řešení od společnosti Google \cite{googlemaps}. V~aplikaci používám zejména zobrazení nabídek v~mapě s~možností otevření informačního okna s~detailem nabídky, převod z~adresy na souřadnice a zpět, zobrazování kruhů v~mapě a další.

\subsection{Technologie na serverové straně}
\subsubsection{Python}
Použití jazyka Python plyne ze zadání. Python je moderní programovací jazyk. Je univerzální – pohání weby i rakety. Dobře se čte a dá se velice rychle naučit. Je skvělý pro výuku programování \cite{python-cz}.

\subsubsection{Django}
Po domluvě s~vedoucím práce jsem se rozhodl pro použití frameworku Django. Django byl původně vytvořen v~novinářském prostředí v~roce 2003 a později v~roce 2005 byl vydán jako open source \cite{djangobook, django-cr}. Nejdůležitější částí frameworku Django je architektura Model-View-Template, která je popsána v~následující kapitole.

\subsubsection{Architektura Model-View-Template (MVT)}
MVT je architektura odvozená od architektury \textbf{Model-View-Controller}. V~architektuře MVT je vynechána vrstva Controller\footnote{Vrstva zprostředkovávající interakci mezi vrstvami Model a View/View-Template}, o~kterou se stará framework Django sám, a která je nahrazena kombinací vrstev View a Template.

\paragraph*{Model}
Model je datový a funkční základ aplikace. Obsahuje definici modelových a servisních tříd.
\paragraph*{View}
V~kontextu frameworku Django je \textbf{view} neboli \textbf{pohled} vrstva stojící mezi vrstvami model a template. Úkolem je získávání dat od uživatele, případně od vrstvy model a předávání těchto dat do šablon. Pohled obsahuje data, která mají být zobrazena, ale nedefinuje, jak mají být zobrazena. Každý pohled je spojen s~konkrétním URL.
\paragraph*{Template}
Vrstva \textbf{template}, neboli \textbf{šablona}, popisuje, jak jsou data zobrazena uživateli. Může se jednat o~reprezentaci v~HTML nebo například reprezentaci ve formátu JSON.