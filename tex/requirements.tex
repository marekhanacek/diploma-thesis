\chapter{Požadavky na aplikaci}
\label{requirements}

Aplikace lidem umožní zadávat nabídky, respektive poptávky, na výměnu určitého obnosu peněz z~jedné měny do druhé, fyzicky v~okolí jejich výskytu a na setkání se s~případným protějškem domluvit.

Následující dvě podkapitoly popisují požadavky na aplikaci z~pohledu funkčních\footnote{Požadavky kladoucí omezení na funkčnost a logiku fungování systému.} a nefunkčních\footnote{Požadavky, které kladou omezení na design a provedení.} požadavků.

Požadavky vychází jak ze zadání, tak z~analýzy podobných webových služeb z~kapitoly \ref{analyza}.
\section{Funkční požadavky}
\begin{itemize}
    \item[\textbf{F1}] Nabídka pro směnu peněz obsahuje:
        \begin{itemize}
            \item Měny, které budou směněny.
            \item Částky v~jednotlivých měnách.
            \item Adresu a poloměr pro vyhledávání.
            \item Komentář k~nabídce.
            \item Hodnocení uživatele za dokončenou nabídku.
        \end{itemize}
    \item[\textbf{F2}] Hodnocení nabídky/uživatele:
        \begin{itemize}
            \item Obsahuje hodnocení pomocí 1 až 5 hvězdiček.
            \item Obsahuje textový komentář.
            \item Hodnocení protistrany je vidno až po uplynutí ochranné lhůty nebo ihned po zadání vlastního hodnocení.
            \item Ostatní uživatelé vidí hodnocení až po zadání hodnocení obou stran nebo po uplynutí ochranné lhůty.
            \item Je nutné, aby měl uživatel dostatečný čas na přidání hodnocení, a současně aby se hodnocení v~budoucnu zobrazilo. Proto je ochranná lhůta stanovena na 7 dní.
        \end{itemize}
    \item[\textbf{F3}] Vyhledávání nabídky:
        \begin{itemize}
            \item Vyhledávat nabídky může přihlášený i nepřihlášený uživatel.
            \item Obsahuje měny a interval obnosu měn, které budou směněny.
            \item Lokace nebude zadávána přímo, ale bude brána ta, která je předvyplněna v~profilu uživatele\footnote{V případě nepřihlášeného uživatele se nejedná o~profil uživatele, ale o~data, která jsou uložena v~rámci relace.}.
        \end{itemize}
    \item[\textbf{F4}] Vytváření nabídky:
        \begin{itemize}
            \item Obsahuje měny a částky, které budou směněny.
            \item Uživatel nezadává \textit{částku do}. Ta bude počítána dle aktuálního středového kurzu.
            \item Obsahuje lokaci, která bude předvyplněna dle adresy z~profilu uživatele.
        \end{itemize}
    \item[\textbf{F5}] Přihlášení uživatele:
        \begin{itemize}
            \item Uživatelé se budou přihlašovat pouze přes Facebook API.
            \item Klasická registrace pomocí uživatelského jména a hesla není možná\footnote{Tato registrace, pro aplikaci sice smysl dává, ale v~rozsahu této práce, po domluvě s~vedoucím práce, nebude implementována.}.
        \end{itemize}
    \item[\textbf{F6}] Profil uživatele obsahuje:
        \begin{itemize}
            \item Jméno uživatele.
            \item Čas posledního přihlášení uživatele.
            \item Hodnocení uživatele.
            \item Profilovou fotku ze sociální sítě Facebook.
            \item Adresu používanou pro směnu peněz.
            \item Poloměr pro zadávání a vyhledávání nabídek.
        \end{itemize}
    \item[\textbf{F7}] Nepřihlášený uživatel je oprávněn:
        \begin{itemize}
            \item Zobrazit seznam/mapu nabídek.
            \item Filtrovat nabídky dle lokality.
            \item Filtrovat nabídky dle zvolených měn a částek.
            \item Řadit nabídky v~podobě seznamu dle definovaných kritérii.
            \item Prohlížet informativní stránky webu.
        \end{itemize}
\end{itemize}


\section{Nefunkční požadavky}

\begin{itemize}
    \item[\textbf{N1}] Dostupnost přes webové rozhraní:
        \begin{itemize}
            \item Web bude přístupný na internetu pomocí webového prohlížeče.
        \end{itemize}
    \item[\textbf{N2}] Dostupnost přes webové API:
        \begin{itemize}
            \item Bude přístupné API pro plánovanou mobilní aplikaci.
        \end{itemize}
    \item[\textbf{N3}] Responzivní design:
        \begin{itemize}
            \item Web bude optimalizován pro mobilní zařízení s~minimální šířkou displeje 360 px. Pro dosažení větší přehlednosti lze pro displeje s~menším rozlišením skrýt prvky, které nejsou nezbytně nutné.
        \end{itemize}
    \item[\textbf{N4}] Optimalizace v~prohlížeči:
        \begin{itemize}
            \item Web bude testován v~následujících webových prohlížečích:
                \begin{itemize}
                    \item Google Chrome.
                    \item Internet Explorer od verze 8.0.
                    \item Mozilla Firefox.
                    \item Android mobile.
                    \item Safari.
                \end{itemize}
        \end{itemize}
\end{itemize}