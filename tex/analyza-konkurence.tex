\chapter{Analýza podobných webů}

Otázka zní: Dělá někdo něco stejné? Odpoveď zní ne, ale existují webové aplikace, které spracovávají některou námi žádanou funkcionalitu. Proto si v moji diplomové práci vyhledám více informací o následujících webových aplikacích:
\begin{itemize}
	\item \textbf{BlaBlaCar} (\url{https://www.blablacar.cz}) – Velmi podobná problematika, jen v jiném kabátu. Obsahuje vyhledávání nabídek a hodnocení uživatelů, kteří zadávají své jízdy ke spolujízdě, případně se k těmto spolujízdám přihlašují.

	\item \textbf{Couchsurfing} (\url{https://www.couchsurfing.com}) – Komunita lidí, kteří rádi cestují a chtějí poznávat svět, ale nechtějí platnit velké množství peněz za hotely. Stačí, když sdílíš svůj "couch" a někdo, kdo chce navštívit dané mesto, se může na pár dní uchýlit. Pro nás jsou důležitými prvky hodnocení uživatelů, jejich ověření a lokalizované vyhledávání nabídek.

	\item \textbf{SReality.cz} (\url{https://www.sreality.cz}) – Realitní agentura, která je pro nás zajímavá čistým a přehledným vzhledem svojí webové aplikace a hlavně rychlým vyhledáváním nabídek přímo na mapě.

	\item \textbf{Aukro} (\url{http://aukro.cz}) – Prodej výrobku, který zaujme zpracovaním hodnocení uživatelů/prodejců.

	\item \textbf{Zonky} (\url{https://zonky.cz}) – Bankovní a nebankovní půjčky. Jinak řečeno polidštění jinak nepříjemných sezení v bance. S mou prací mají spoločnou přímou pomoc/jednání mezi lidmi a tím pádem možnost výhodnějších nabídek.
\end{itemize}

\newpage
\section{BlaBlaCar \cite{blablacar}}
\label{analyza:blablacar}

Tak, jak na své úvodní stránce BlaBlaCar uvádí:

\begin{center}
\textit{BlaBlaCar je přední světová komunita spolujízdy, která spojuje řidiče a cestující na stejné trase, a umožňuje tak levné meziměstské cestování.}
\end{center}

Na základě tohoto jednoduchého principu si lidé mohou přisednout k někomu jako spolucestující. Tímto způsobem se tak dostanou k častokrát levnější a pohodlnější formě přepravy. Pro řidiče je výhodou částečné proplacení jízdy (pohonných hmot) těmito spolucestujícími.\\

Postupně projdu každou, pro mě důležitou stránku, webu BlaBlaCar a u každé takové stránky definuji seznamy pozitivních a negativních vlastností.

\subsection{Hlavní stránka}
Viz obrázek \ref{fig:blablacar:homepage}.
\begin{figure}[h]
    \centering
    \includegraphics[width=1.0\textwidth]{media/blablacar/homepage.png}
    \caption{Hlavní stránka webu BlaBlaCar.cz}
    \label{fig:blablacar:homepage}
\end{figure}
\subsubsection*{Pozitiva}
\begin{itemize}
    \item[+] \textbf{Přehlednost} -- Na stránce jsou výrazně viditelné dva nejdůležitější typy úkonů a to \textit{Nabídnutí jízdy} a \textit{Vyhledávání jízdy}. I v mém návrhu budu brát ohled na to, aby každá důležitá akce měla nejvyšší prioritu.
    \item[+] \textbf{Ostatní možnosti} -- V pravém horním rohu je dostupný profil uživatele stejně jako nové upozornění na události týkajících se uživatelského profilu.
    \item[+] \textbf{Jak to funguje} -- Každému uživateli na první pohled nemusí být jasné, o co se přesně jedná. Proto web BlaBlaCar.cz na své úvodní stránce uvádí postup jak se na spolujízdu registrovat.
    \item[+] \textbf{Oblíbené trasy} -- Seznam tří uživateli nejoblíbenějších tras.
\end{itemize}
\subsubsection*{Negativa}
\begin{itemize}
    \item[-] \textbf{Žádná negativa na této stránce nevidím.}
\end{itemize}


%%%%%%%%%%%%%%%%%%%%%%%%%%%%%%%%%%%%%%%%%%%%%%%%%%%%%%%%%%%%%%%%%%%%%%%%%%%%%%%%%%%%%%%%%%%%%%%%%%%%%%%%%%%%%%%%%%%%%%%%

\newpage
\subsection{Vyhledávání jízdy}
Viz obrázek \ref{fig:blablacar:search}.
\begin{figure}[h]
    \centering
    \includegraphics[width=1.0\textwidth]{media/blablacar/search.png}
    \caption{Vyhledávání nabídek na webu BlaBlaCar.cz}
    \label{fig:blablacar:search}
\end{figure}
\subsubsection*{Pozitiva}
\begin{itemize}
    \item[+] \textbf{Informativnost} -- Hned na první pohled uživatel vidí všechny relevantní informace: čas, cenu, počet volných míst, délku trasy a hodnocení daného řidiče.
    \item[+] \textbf{Filtry} -- Možnost filtrovaní požadavků na základě zkušeností řidiče a pohodlí auta považuji za nejzajímavější.
    \item[+] \textbf{Možnost řazení} -- Seřazení nabídek podle ceny a času odjezdu je určitě velmi vítaná a potřebná vlastnost.
\end{itemize}
\subsubsection*{Negativa}
\begin{itemize}
    \item[-] \textbf{Nedostupnost profilu řidiče na jeden klik} -- Na první pohled očekávaná funkcionalita (existence předělu mezi cestou a profilem řidiče), při které by se uživatel po kliknutí myší na pravou část nabídky dostal na bližší informace o jízdě a po kliknutí na levou část nabídky dostal na profil řidiče. Tohoto problému se při návrhu uživatelského rozhraní budu snažit vyvarovat.
\end{itemize}


%%%%%%%%%%%%%%%%%%%%%%%%%%%%%%%%%%%%%%%%%%%%%%%%%%%%%%%%%%%%%%%%%%%%%%%%%%%%%%%%%%%%%%%%%%%%%%%%%%%%%%%%%%%%%%%%%%%%%%%%

\newpage
\subsection{Detail jízdy}
Viz obrázek \ref{fig:blablacar:detail}.
\begin{figure}[h]
    \centering
    \includegraphics[width=1.0\textwidth]{media/blablacar/detail.png}
    \caption{Detail jízdy na webu BlaBlaCar.cz}
    \label{fig:blablacar:detail}
\end{figure}
\subsubsection*{Pozitiva}
\begin{itemize}
    \item[+] \textbf{Harmonogram} -- Graficky velmi pěkně řešený přehled celé jízdy a spolucestujících včetně časů odjezdů a příjezdů.
    \item[+] \textbf{Spolucestující} -- Možnost vidět kdo s vámi cestuje je vítaná, jelikož s někým se rádi svezete a někomu se naopak raději vyhnete.
    \item[+] \textbf{Podrobnosti} -- Tímto se řidič vyhne nepříjemnostem s velkým počtem zavazadel a uživatel s delšími zajížďkami řidiče.
\end{itemize}
\subsubsection*{Negativa}
\begin{itemize}
    \item[-] \textbf{Žádná negativa na této stránce nevidím.}
\end{itemize}


%%%%%%%%%%%%%%%%%%%%%%%%%%%%%%%%%%%%%%%%%%%%%%%%%%%%%%%%%%%%%%%%%%%%%%%%%%%%%%%%%%%%%%%%%%%%%%%%%%%%%%%%%%%%%%%%%%%%%%%%

\newpage
\subsection{Nabídnutí jízdy}
Viz obrázky \ref{fig:blablacar:offer} a \ref{fig:blablacar:offer2}.
\begin{figure}[h]
    \centering
    \includegraphics[width=1.0\textwidth]{media/blablacar/offer.png}
    \caption{Zadání nové jízdy na webu BlaBlaCar.cz - Harmonogram}
    \label{fig:blablacar:offer}
\end{figure}
\begin{figure}[h]
    \centering
    \includegraphics[width=1.0\textwidth]{media/blablacar/offer2.png}
    \caption{Zadání nové jízdy na webu BlaBlaCar.cz - Podrobnosti}
    \label{fig:blablacar:offer2}
\end{figure}
\subsubsection*{Pozitiva}
\begin{itemize}
    \item[+] \textbf{Krok za krokem} -- Uživatel postupně prochází všemi důležitými aspekty nabídky jízdy.
    \item[+] \textbf{Přijatelné UI} -- Všechno je na svém místě a výrazně odlišené od ostatních položek.
    \item[+] \textbf{Doporučená cena} -- Automatické vyplnění ceny na základě ostatních nabídek a vzdálenosti.
    \item[+] \textbf{Mapa} -- Mapa i celkové shrnutí vzdálenosti a trvání jízdy. Speciálně zajímavým prvkem se mi jeví množství emisí vypuštěných do ovzduší.
\end{itemize}
\subsubsection*{Negativa}
\begin{itemize}
    \item[-] \textbf{Nevidím}
\end{itemize}


%%%%%%%%%%%%%%%%%%%%%%%%%%%%%%%%%%%%%%%%%%%%%%%%%%%%%%%%%%%%%%%%%%%%%%%%%%%%%%%%%%%%%%%%%%%%%%%%%%%%%%%%%%%%%%%%%%%%%%%%

\newpage
\subsection{Profil uživatele}
Viz obrázek \ref{fig:blablacar:profile}.
\begin{figure}[h]
    \centering
    \includegraphics[width=1.0\textwidth]{media/blablacar/profile.png}
    \caption{Profil uživatele na webu BlaBlaCar.cz}
    \label{fig:blablacar:profile}
\end{figure}
\subsubsection*{Pozitiva}
\begin{itemize}
    \item[+] \textbf{Informativnost} -- Recenze, typ auta, počet nabídnutých jízd. Jednoduše se dozvíme vše co potřebujeme bez nutnosti přecházet na další stránku.
    \item[+] \textbf{Ověření} -- Různé úrovně ověření každého uživatele. Tato funkcionalita se mi líbí a zvážím její přidání do mé aplikace.
\end{itemize}
\subsubsection*{Negativa}
\begin{itemize}
    \item[-] \textbf{Nevidím}
\end{itemize}


%%%%%%%%%%%%%%%%%%%%%%%%%%%%%%%%%%%%%%%%%%%%%%%%%%%%%%%%%%%%%%%%%%%%%%%%%%%%%%%%%%%%%%%%%%%%%%%%%%%%%%%%%%%%%%%%%%%%%%%%

\newpage
\subsection{Shrnutí}
Pro účel aplikace jsou nejdůležitější dvě stránky, které BlaBlaCar poskytuje a to: \textbf{Vyhledávání jízdy} a \textbf{Detail jízdy}.

Při návrhu uživatelského rozhraní se zaměřím na poskytnutí možnosti filtrování a také seřazení nabídek. Zakomponuji rozdílnost kliknutí na uživatele, která uživatele dostane na jeho profil, resp. samotné nabídky, která nás přesměruje na bližší informace o dané nabídce.

I napříč tomu, že v moji aplikaci neexistují \textbf{spolucestující}, tak si beru příklad z této vlastnosti BlaBlaCar a při detailech nabídky zvážím výpis historie s daným uživatelem, která slouží velmi podobnému účelu.

Hodnocení uživatelů je řešeno pomocí až pěti hvězdiček.
\newpage
\section{Couchsurfing \cite{couchsurfing}}
\label{analyza:couchsurfing}

Ubytování bývá drahé a proto existují lidé, kteří na pár nocí vypůjčují svůj gauč a tím pomohou někomu poznat jejich město a zemi. Pro cestující z toho plyne ještě jeden pozitivní dopad -- skvělé informace, které by se nikde jinde nedozvěděli, a to přímo od lokálního člověka.\\

\subsection{Hlavní stránka}
\begin{figure}[h]
    \centering
    \includegraphics[width=1.0\textwidth]{media/couchsurfing/homepage.png}
    \caption{Hlavní stránka webu couchsurfing.cz}
    \label{fig:couchsurfing:homepage}
\end{figure}
\subsubsection*{Pozitiva}
\begin{itemize}
    \item[+] \textbf{Přehlednost} -- Všechno důležité na jednom místě a přehledně oddělené.
    \item[+] \textbf{Vyhledávací okno} -- Rychlé vyhledávací okno v horní části stránky.
\end{itemize}
\subsubsection*{Negativa}
\begin{itemize}
    \item[-] \textbf{Stránka je dlouhá} -- Pro zobrazení veškerého obsahu je potřeba dlouho točit kolečkem myši.
\end{itemize}


%%%%%%%%%%%%%%%%%%%%%%%%%%%%%%%%%%%%%%%%%%%%%%%%%%%%%%%%%%%%%%%%%%%%%%%%%%%%%%%%%%%%%%%%%%%%%%%%%%%%%%%%%%%%%%%%%%%%%%%%

\newpage
\subsection{Vyhledávání ubytovaní}
\begin{figure}[h]
    \centering
    \includegraphics[width=1.0\textwidth]{media/couchsurfing/search.png}
    \caption{Vyhledávání ubytování na webu couchsurfing.cz}
    \label{fig:couchsurfing:search}
\end{figure}
\subsubsection*{Pozitiva}
\begin{itemize}
    \item[+] \textbf{Jednoduchost}
    \item[+] \textbf{Filtry} -- Všechno důležité s možností rozšířeného filtru.
    \item[+] \textbf{Status} -- Ověření uživatelé jsou jasně viditelní.
\end{itemize}
\subsubsection*{Negativa}
\begin{itemize}
    \item[-] \textbf{Nevidím}
\end{itemize}


%%%%%%%%%%%%%%%%%%%%%%%%%%%%%%%%%%%%%%%%%%%%%%%%%%%%%%%%%%%%%%%%%%%%%%%%%%%%%%%%%%%%%%%%%%%%%%%%%%%%%%%%%%%%%%%%%%%%%%%%

\newpage
\subsection{Profil}
\begin{figure}[h]
    \centering
    \includegraphics[width=1.0\textwidth]{media/couchsurfing/profile.png}
    \caption{Profil uživatele na webu couchsurfing.cz}
    \label{fig:couchsurfing:profile}
\end{figure}
\subsubsection*{Pozitiva}
\begin{itemize}
    \item[+] \textbf{Informativnost} -- Všechno kompaktně na jednom místě a přehledně.
    \item[+] \textbf{About me} -- To zda mi daný člověk bude vyhovovat častokrát zjistíme už z toho jak dokáže popsat sám sebe. Určitě vítaná funkcionalita.
\end{itemize}
\subsubsection*{Negativa}
\begin{itemize}
    \item[-] \textbf{Nevidím}
\end{itemize}


%%%%%%%%%%%%%%%%%%%%%%%%%%%%%%%%%%%%%%%%%%%%%%%%%%%%%%%%%%%%%%%%%%%%%%%%%%%%%%%%%%%%%%%%%%%%%%%%%%%%%%%%%%%%%%%%%%%%%%%%

\newpage
\subsection{Shrnutí}
Celkově se mi webová aplikace Couchsurfing jeví velmi dobře řešená.\\

Důležitým mottem, s kterým buduji webovou aplikaci, je být \textbf{přehledný}. Každá podstránka by měla obsahovat všechno co je potřeba a nic víc. Zajímavou myšlenkou se mi jeví existence \textbf{verifikovaných uživatelů}, která by v této práci vytvořila důvěru a určitou záruku korektního jednání při výměně peněz. Nevýhodu vidím ve velmi dlouhém provedení úvodní stránky, čehož se budu snažit vyvarovat a tím pádem ulehčit uživateli orientaci na stránce.
\newpage
\section{SReality.cz}
\label{chap:sreality}
Realitní agentura SReality si vytvořila webovou aplikaci, která má ulehčovat uživatelům výběr domu či bytu, ve kterém stráví následující roky svého života. Takové rozhodnutí bývá velmi náročné a proto od sreality.cz očekávam jasné podání informací a hlavně přijatelné uživatelské prostředí. Toto očekávání je znásobené ještě tím, že uživatelé, kteří navštíví danou stránku, mohou být z různých věkových skupin.\\

\subsection{Home page}
\begin{figure}[h]
    \centering
    \includegraphics[width=1.0\textwidth]{media/sreality/homepage.png}
    \caption{Hlavní stránka webu sreality.cz}
    \label{fig:sreality:homepage}
\end{figure}
\subsubsection*{Pozitiva}
\begin{itemize}
    \item[+] \textbf{Jednoduchost} -- Bílé prostředí, které obsahuje pouze černé nápisy a červené obrysové nákresy. Velmi jednoduché a přehledné řešení.
    \item[+] \textbf{Poslední a oblíbené vyhedávání} -- Speciální možnosti, které ulehčí hledání uživatelům, kteří již na dané stránce byli a možná si už i nějaké nabídky vybrali.
\end{itemize}
\subsubsection*{Negativa}
\begin{itemize}
    \item[-] \textbf{Nevidím}
\end{itemize}


%%%%%%%%%%%%%%%%%%%%%%%%%%%%%%%%%%%%%%%%%%%%%%%%%%%%%%%%%%%%%%%%%%%%%%%%%%%%%%%%%%%%%%%%%%%%%%%%%%%%%%%%%%%%%%%%%%%%%%%%

\newpage
\subsection{Vyhledávání realit}
\begin{figure}[h]
    \centering
    \includegraphics[width=1.0\textwidth]{media/sreality/detail.png}
    \caption{Detail nabídky na webu sreality.cz}
    \label{fig:sreality:detail}
\end{figure}
\begin{figure}[h]
    \centering
    \includegraphics[width=1.0\textwidth]{media/sreality/detail-big-map.png}
    \caption{Detail nabídky na webu sreality.cz - větší mapa}
    \label{fig:sreality:detail-big-map}
\end{figure}
\subsubsection*{Pozitiva}
\begin{itemize}
    \item[+] \textbf{Mapa} -- Krásne propojení seznam map\footnote{mapy.cz a i sreality.cz jsou projekty seznam.cz} a vyhledávání. Navíc je hned vidět jak daleko se nachází zástávka městské hromadné dopravy nebo nejlibžší supermarket a další.
    \item[+] \textbf{Lišta s náhledem nabídky} -- Informativní popis všech vlastností dané nabídky.
    \item[+] \textbf{Zvětšení/zmenšení} -- Uživatel má možnost zvětšit, nebo zmenšit mapu a tím pádem vidět méně či naopak více z obsahu.
\end{itemize}
\subsubsection*{Negativa}
\begin{itemize}
    \item[-] \textbf{Nemožnost čistého náhledu bez mapy} -- Při užším a delším vybírání je možné, že lokalitu mám jako uživatel dobře zmapovanou a tím pádem nepotřebuji vidět mapu (ani v malé formě) na levé straně obrazovky.
\end{itemize}


%%%%%%%%%%%%%%%%%%%%%%%%%%%%%%%%%%%%%%%%%%%%%%%%%%%%%%%%%%%%%%%%%%%%%%%%%%%%%%%%%%%%%%%%%%%%%%%%%%%%%%%%%%%%%%%%%%%%%%%%

\newpage
\subsection{Shrnutí}
Celkově se mi webová aplikace sreality.cz jeví jako velmi dobře řešená.

Nejzajímavějším aspektem SReality.cz se mi jeví přítomnost \textbf{mapy}, která ulehčuje a urychluje výběr. Tuto vlastnost určite chci zakomponovat do své webové aplikace. Forma přepínání mezi mapou a obsahem je řešená \textbf{jednoduše} a uživatelsky přijatelně. Podle mě je však dobré se \textbf{vyvarovat nutnosti vždy zobrazovat mapu} a ponechat tuto funkcionalitu jen jako možnost.

Hodnocení uživalů zde není nijak řešené.
\newpage
\section{Aukro}

\label{chap:aukro}

Webová aplikace, která se jak napovídá název nezabývá jen aukcemi, ale také normálním prodejem zboží. Protože je to ve své podstatě e-shop, tak od něj očekávám dobře zvolené kategorie, detaily v popise zboží a v případě aukra i výrazné zobrazení hodnocení prodejců zboží.

\subsection{Home page}
\begin{figure}[h]
    \centering
    \includegraphics[width=1.0\textwidth]{media/aukro/home.png}
    \caption{Hlavní stránka webu aukro.cz}
    \label{fig:aukro:home}
\end{figure}
\subsubsection*{Pozitiva}
\begin{itemize}
    \item[+] \textbf{Co hledáte?} -- Nejdůležitější prvek (vyhledávání) úvodní stránky je dostatečně velký a je na místě, kde ho uživatelé očekávájí.
    \item[-] \textbf{Kategorie} -- Dobré řešení a při najetí myší se kromě podkategorií zobrazí i populární kategorie a jedna speciální nabídka zboží. Myslím si, že toto chování způsobí, že si nějaká podskupina uživatelů nakoupí i něco, co předtím nechtěli, nebo to nebylo jejich prioritou.
\end{itemize}
\subsubsection*{Negativa}
\begin{itemize}
    \item[-] \textbf{Poutač} -- Velký poutač na speciální akce aukra. Pokud chce uživatel vidět nějaké zboží, které mu vybralo aukro, nebo naposledy shlédnuté zboží, tak musí přejít kolečkem myši dlouhý kus stránky.
\end{itemize}


%%%%%%%%%%%%%%%%%%%%%%%%%%%%%%%%%%%%%%%%%%%%%%%%%%%%%%%%%%%%%%%%%%%%%%%%%%%%%%%%%%%%%%%%%%%%%%%%%%%%%%%%%%%%%%%%%%%%%%%%

\newpage
\subsection{Vyhledávání zboží}
\begin{figure}[h]
    \centering
    \includegraphics[width=1.0\textwidth]{media/aukro/search.png}
    \caption{Vyhledávání zboží na webu aukro.cz}
    \label{fig:aukro:search}
\end{figure}
\subsubsection*{Pozitiva}
\begin{itemize}
    \item[+] \textbf{Cena i s dopravou} -- Na stránce je přímo zobrazená i cena s dopravou. Uživatel cítí určitou transparentnost a nemusí si tyto informace vyhledávat sám.
\end{itemize}
\subsubsection*{Negativa}
\begin{itemize}
    \item[-] \textbf{Málo informací o zboží} -- O konkrétním zboží se člověk kromě názvu moc nedozví. Místo na alespoň pár detailů tam přitom existuje.
    \item[-] \textbf{Málo nabídek} -- Dvě nabídky zaberou celou stránku ve výchozím nastavení.
    \item[-] \textbf{Chaotické rozhraní} -- Rozhraní je chaotické a neuspořádané.
\end{itemize}


%%%%%%%%%%%%%%%%%%%%%%%%%%%%%%%%%%%%%%%%%%%%%%%%%%%%%%%%%%%%%%%%%%%%%%%%%%%%%%%%%%%%%%%%%%%%%%%%%%%%%%%%%%%%%%%%%%%%%%%%

\newpage
\subsection{Detail zboží}
\begin{figure}[h]
    \centering
    \includegraphics[width=1.0\textwidth]{media/aukro/detail.png}
    \caption{Detail zboží na webu aukro.cz}
    \label{fig:aukro:detail}
\end{figure}
\subsubsection*{Pozitiva}
\begin{itemize}
    \item[+] \textbf{Nevidím}
\end{itemize}
\subsubsection*{Negativa}
\begin{itemize}
    \item[-] \textbf{Neobsahuje moc detailů} -- Detaily sa nacházejí až po delším kuse stránky.
    \item[-] \textbf{Hodnocení prodejců} -- Sice se hodnocení prodejců na stránce nachází, ale výrazně je viditelnější až po rozkliknutí.
    \item[-] \textbf{Chaotické rozhraní} -- Problém uživatelského rozhraní zůstává.
\end{itemize}


%%%%%%%%%%%%%%%%%%%%%%%%%%%%%%%%%%%%%%%%%%%%%%%%%%%%%%%%%%%%%%%%%%%%%%%%%%%%%%%%%%%%%%%%%%%%%%%%%%%%%%%%%%%%%%%%%%%%%%%%

\newpage
\subsection{Profil prodejce}
\begin{figure}[h]
    \centering
    \includegraphics[width=1.0\textwidth]{media/aukro/profile.png}
    \caption{Profil prodejce na webu aukro.cz}
    \label{fig:aukro:profile}
\end{figure}
\begin{figure}[h]
    \centering
    \includegraphics[width=1.0\textwidth]{media/aukro/profile2.png}
    \caption{Recenze uživatele na webu aukro.cz}
    \label{fig:aukro:profile2}
\end{figure}
\subsubsection*{Pozitiva}
\begin{itemize}
    \item[+] \textbf{Historie hodnocení/komentářů} -- Možnost přečíst si kdo byl jak spokojený s daným prodejcem.
\end{itemize}
\subsubsection*{Negativa}
\begin{itemize}
    \item[-] \textbf{Nepřehlednost} -- Problémy s rozhraním se projevují i na této stránce. Nedá se vytvořit si závěry bez dlouhého kroucení kolečkem myši.
\end{itemize}


%%%%%%%%%%%%%%%%%%%%%%%%%%%%%%%%%%%%%%%%%%%%%%%%%%%%%%%%%%%%%%%%%%%%%%%%%%%%%%%%%%%%%%%%%%%%%%%%%%%%%%%%%%%%%%%%%%%%%%%%

\newpage
\subsection{Shrnutí}
Úpřímě řečeno jsem při rešeršování aukro.cz nabudil pocit, že chci z dané webové aplikace co nejdříve odejít. Z mých očekávání se nesplnilo nic a tak se musím poučit z chyb, kterých se dopustili. Konkrétně je to:
\begin{itemize}
    \item \textbf{Chaotické rozhraní}
    \item \textbf{Nedetailnost}
    \item \textbf{Málo poskytnutých informací}
\end{itemize}

Z mého pohledu se jeví být zajímavou funkcí \textbf{historie komentářů}, kdy si uživatel může vytvořit názor o prodejci . Chci však, aby mé rozhraní bylo přehledné a určitě aby poskytovalo všechny potřebné detaily. To pro mě znamená: jaká je daná nabídka, jak daleko je vzdálená a jaké hodnocení má prodejce - a to vše na jednom místě.
\newpage
\section{Zonky}
\label{analyza:zonky}

Přímé půjčky od lidí, kteří rádi pomohou. Jinak řečeno -- buďme lidští a vynechejme z~toho banky.

\subsection{Hlavní stránka}
Viz obrázek \ref{fig:zonky:home}.
\begin{figure}[h]
    \centering
    \includegraphics[width=1.0\textwidth]{media/zonky/home.png}
    \caption{Zonky.cz -- Hlavní stránka}
    \label{fig:zonky:home}
\end{figure}
\subsubsection*{Pozitiva}
\begin{itemize}
    \item[+] \textbf{Jednoduchost a přehlednost} -- Existence pouze pár tlačítek a jasná čitelnost všeho, co se na dané stránce nachází.
\end{itemize}
\subsubsection*{Negativa}
\begin{itemize}
    \item[-] \textbf{Zvuk} -- Velmi rušivý element, který se zvýší v~momentě, kdy má uživatel puštěnou na svém zařízení hudbu.
\end{itemize}


%%%%%%%%%%%%%%%%%%%%%%%%%%%%%%%%%%%%%%%%%%%%%%%%%%%%%%%%%%%%%%%%%%%%%%%%%%%%%%%%%%%%%%%%%%%%%%%%%%%%%%%%%%%%%%%%%%%%%%%%

\newpage
\subsection{Zlevnovač}
Viz obrázek \ref{fig:zonky:zlevnovac}. % a \ref{fig:zonky:zlevnovac2}.
\begin{figure}[h]
    \centering
    \includegraphics[width=1.0\textwidth]{media/zonky/zlevnovac.png}
    \caption{Zonky.cz -- Zlevňovač}
    \label{fig:zonky:zlevnovac}
\end{figure}
%\begin{figure}[h]
%    \centering
%    \includegraphics[width=1.0\textwidth]{media/zonky/zlevnovac2.png}
%    \caption{Zlevňovač na webu Zonky.cz 2}
%    \label{fig:zonky:zlevnovac2}
%\end{figure}
\subsubsection*{Pozitiva}
\begin{itemize}
    \item[+] \textbf{Jednoduchost a přehlednost} -- Opět jednoduché uživatelské rozhraní, ve kterém se nachází jen to potřebné.
    \item[+] \textbf{Vyhodnocení} -- Rychlý náhled na to, kolik uživatel ušetří, podaný pěknou a jasnou formou.
\end{itemize}
\subsubsection*{Negativa}
\begin{itemize}
    \item[-] \textbf{Zvuk}.
\end{itemize}


%%%%%%%%%%%%%%%%%%%%%%%%%%%%%%%%%%%%%%%%%%%%%%%%%%%%%%%%%%%%%%%%%%%%%%%%%%%%%%%%%%%%%%%%%%%%%%%%%%%%%%%%%%%%%%%%%%%%%%%%

\newpage
\subsection{Tržiště}
Viz obrázek \ref{fig:zonky:marketplace}.
\begin{figure}[h]
    \centering
    \includegraphics[width=1.0\textwidth]{media/zonky/marketplace.png}
    \caption{Zonky.cz -- Tržiště}
    \label{fig:zonky:marketplace}
\end{figure}
\subsubsection*{Pozitiva}
\begin{itemize}
    \item[+] \textbf{Informativnost} -- Krátký popis, rating\footnote{Rating určuje míru rizika.}, úrok a doba splácení je vše co uživatel na první pohled potřebuje.
    \item[+] \textbf{Filtry a řazení} -- Uživatel má šanci si rychle zobrazit nabídky, které ho zajímají.
\end{itemize}
\subsubsection*{Negativa}
\begin{itemize}
    \item[-] \textbf{Žádná negativa na této stránce nejsou pozorována.}
\end{itemize}


%%%%%%%%%%%%%%%%%%%%%%%%%%%%%%%%%%%%%%%%%%%%%%%%%%%%%%%%%%%%%%%%%%%%%%%%%%%%%%%%%%%%%%%%%%%%%%%%%%%%%%%%%%%%%%%%%%%%%%%%

\newpage
\subsection{Návodné a informativní stránky}
Viz obrázky \ref{fig:zonky:info}, \ref{fig:zonky:info2} a \ref{fig:zonky:info3}.
\begin{figure}[h]
    \centering
    \includegraphics[width=1.0\textwidth]{media/zonky/info.png}
    \caption{Zonky.cz -- Když lidé půjčují lidem}
    \label{fig:zonky:info}
\end{figure}
\begin{figure}[h]
    \centering
    \includegraphics[width=1.0\textwidth]{media/zonky/info2.png}
    \caption{Zonky.cz -- Jak funguje půjčování}
    \label{fig:zonky:info2}
\end{figure}
\begin{figure}[h]
    \centering
    \includegraphics[width=1.0\textwidth]{media/zonky/info3.png}
    \caption{Zonky.cz -- O~lidech}
    \label{fig:zonky:info3}
\end{figure}
\subsubsection*{Pozitiva}
\begin{itemize}
    \item[+] \textbf{Uživatelská přívětivost} -- Informace jsou podány snadně, jednoduše a přehledně. Nic není zbytečně rozsáhlé.
\end{itemize}
\subsubsection*{Negativa}
\begin{itemize}
    \item[-] \textbf{Žádná negativa na těchto stránkách nejsou pozorována.}
\end{itemize}


%%%%%%%%%%%%%%%%%%%%%%%%%%%%%%%%%%%%%%%%%%%%%%%%%%%%%%%%%%%%%%%%%%%%%%%%%%%%%%%%%%%%%%%%%%%%%%%%%%%%%%%%%%%%%%%%%%%%%%%%

\subsection{Shrnutí}
Zonky.cz deklaruje, že chce pomoci lidem, resp. poskytuje možnost pro lidi pomáhat si navzájem. Stejná věc je očekávána i od aplikace vyvíjené v~této práci.

Pokud má aplikace pomáhat lidem, tak uživatelské rozhraní musí být příjemné na používání (to znamená \textbf{jednoduché a přehledné}). Stejně jako je to v~případě Zonky. V~každém zákoutí Zonky se k~uživateli dostává \textbf{dostatečné množství informací} (není zahlcen). \textbf{Přítomnost zvuku} je velmi odrazující.
