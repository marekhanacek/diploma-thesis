\section{Nasazení aplikace}
\label{deploy}

Aplikace byla nasazena do cloudu pomocí služby Heroku.com \cite{heroku}. Nasazení probíhalo pomocí vzdáleného repozitáře systému Git ve službě Heroku.com\footnote{Přidáním vzdáleného heroku repozitáře do lokálního Git repozitáře a následně pomocí příkazu \texttt{git push heroku master}.}. K~úspěšnému nasazení aplikace bylo potřeba vytvořit tyto soubory v~kořenovém adresáři projektu:
\begin{itemize}
    \item \texttt{Procfile} -- Definuje, spustitelný soubor aplikace.
    \item \texttt{requirements.txt} -- Obsahuje seznam závislostí aplikace. Při nasazování aplikace jsou tyto závislosti instalovány pomocí systému PIP \cite{pip}. Seznam závislostí je uveden v~ukázce kódu \ref{code:requirements}.
    \item \texttt{runtime.txt} -- Definuje použitou verzi jazyka Python. V~tomto konkrétním případě je to \texttt{python-3.6.1}.
\end{itemize}

\begin{listing}[h]
\caption{\label{code:requirements}Soubor requirements.txt -- Seznam závislostí aplikace}
\begin{minted}[frame=lines,bgcolor=codebg,fontsize=\footnotesize,linenos,breaklines]{Python}
defusedxml==0.5.0
Django==1.11
djangorestframework==3.6.2
easy-thumbnails==2.4.1
gunicorn==19.7.1
mysqlclient==1.3.7
oauthlib==2.0.2
olefile==0.44
Pillow==4.1.1
PyJWT==1.5.0
python3-openid==3.1.0
pytz==2017.2
requests==2.13.0
requests-oauthlib==0.8.0
six==1.10.0
social-auth-app-django==1.1.0
social-auth-core==1.2.0
whitenoise==3.3.0
\end{minted}
\end{listing}