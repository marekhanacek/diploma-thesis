\section{Testování}
Django framework poskytuje sadu tříd určenou pro testování. Abychom mohli začít testovat je nutné vytvořit testovací data.

\subsection{Testovací data, neboli fixtures}
\begin{sloppypar}
Django při testování ve výchozím stavu vytváří testovací databázi se stejnou strukturou jakou má výchozí databáze. Testovací data pak lze s~testem propojit pomocí souborů ve formátu JSON, XML nebo YAML. V~práci používám formát JSON. Tyto soubory lze vytvořit dvěma způsoby:

\begin{enumerate}
    \item Manuálním vytvořením souboru.
    \item Nebo pomocí příkazu \texttt{python manage.py dumpdata \\--exclude=contenttypes > fixtures/initial\_data.json}, což vytváří testovací data na základě položek uložených v~aktuální databázi. Pomocí parametru \texttt{--exclude} lze definovat položky, které budou vynechány.
\end{enumerate}

Výsledný soubor může pak mít podobu jako ukázka kódu \ref{code:fixtures-json}.

\begin{listing}[h]
\caption{\label{code:fixtures-json}Ukázka testovacích dat ve formátu JSON}
\begin{minted}[frame=lines,bgcolor=codebg,fontsize=\footnotesize,linenos,breaklines]{JSON}
[
  {
    "model": "web.language",
    "pk": 1,
    "fields": {
      "identificator": "en",
      "name": "English"
    }
  },
    {
    "model": "web.offerstatus",
    "pk": 1,
    "fields": {
      "title": "Awaiting acceptance",
      "description": "Awaiting acceptance"
    }
  },
  ...
]
\end{minted}
\end{listing}

Testovací data pak lze do testu přiřadit pomocí proměnné fixtures, jak ukazuje ukázka kódu \ref{code:fixtures-in-code}.

\begin{listing}[h]
\caption{\label{code:fixtures-in-code}Ukázka přiřazení fixtures do testovací třídy}
\begin{minted}[frame=lines,bgcolor=codebg,fontsize=\footnotesize,linenos,breaklines]{Python}
from web.tests import BaseTestCase

class OfferFacadeTests(BaseTestCase):
    fixtures = ['initial_data.json']

    def test_create_offer(self):
        pass
\end{minted}
\end{listing}
\end{sloppypar}

\subsection{Aserce}
Aserce jsou kontroly nad testovacími daty, které musí být splněny při každém spuštění testů. V~Django frameworku to jsou metody s~názvem začínajícím na slovo assert. Například assertEqual, assertIsNone, assertTrue, assertFalse, assertGreater, assertGreaterEqual a další. Více informací o~asercích naleznete v~dokumentaci frameworku \cite{django-testing-tools}.

\subsection{Testování pohledů}
Vytvořil jsem testy pro všechny pohledy v~aplikaci. U~každého pohledu ověřuji správný návratový kód a přítomnost důležitých elementů na stránce. Testuji také chování aplikace v~závislosti na tom, zda je uživatel přihlášen či nikoliv. Netestuji pouze případy, kdy má aplikace skončit úspěchem ale také jsem se snažil psát testy i na ty situace, kdy je požadavek chybný. Testuji také různá chování v~závislosti na zvolené HTTP metodě.

\subsection{Testování servisních tříd}
Vytvořil jsem také všechny důležité testy pro vytvořené servisní třídy.
