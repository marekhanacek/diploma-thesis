\department{Katedra softwarového inženýrství}
\title{Webová aplikace pro směnu peněz mezi lidmi}
\authorGN{Marek} %(křestní) jméno (jména) autora
\authorFN{Hanáček} %příjmení autora
\authorWithDegrees{Bc. Marek Hanáček} %jméno autora včetně současných akademických titulů
\supervisor{Ing. Miroslav Hrončok}
%\acknowledgements{Doplňte, máte-li komu a za co děkovat. V~opačném případě úplně odstraňte tento příkaz.}
\placeForDeclarationOfAuthenticity{V~Praze}
\declarationOfAuthenticityOption{4} %volba Prohlášení (číslo 1-6)
\keywordsCS{Python, Django, směna peněz, webová aplikace}
\keywordsEN{Python, Django, currency exchange, web application}
%\website{http://site.example/thesis} %volitelná URL práce, objeví se v tiráži -- úplně odstraňte, nemáte-li URL práce


\abstractCS{Tato práce je věnována analýze, návrhu a implementaci webové aplikace, a s tím spojeného webového API, která lidem umožní zadávat nabídky, respektive poptávky, na osobní směnu určitého obnosu peněz z~jedné měny do druhé a na takové směně se případně s protějškem domluvit. Na základě analýzy podobných webových služeb jsou definovány požadavky na aplikaci. Na základě analýzy a požadavků bylo navrženo uživatelského rozhraní. Práce je dále věnována návrhu aplikace z~implementačního hlediska a následně implementována ve frameworku Django v programovacím jazyce Python. Na závěr byla aplikace otestována s~uživateli.}
%\abstractCS{V~několika větách shrňte obsah a přínos této práce v~češtině. Po přečtení abstraktu byse čtenář měl mít čtenář dost informací pro rozhodnutí, zda chce Vaši práci číst.}
\abstractEN{This thesis is devoted to the analysis, design and implementation of the web application and the web API, which allows people to bid or demand to personally exchange a certain amount of money from one currency to another. Based on an analysis of similar web services, application requirements are defined. Based on the analysis and requirements, a user interface has been designed. The thesis is also devoted to application design from the implementation point of view and subsequently implemented in Django framework in the Python programming language. Finally, the application was tested with users.}