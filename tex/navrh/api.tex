\pagebreak
\section{Návrh REST API}\label{rest-api-design}
V~následujících podkapitolách jsou uvedeny všechny zdroje spolu s~příslušnou HTTP metodou. Veškeré požadavky i odpovědi serveru jsou ve formátu JSON.

\subsection{GET /api/offer}
Zdroj pro získávání nabídek s~možností filtrace, stránkování a řazení pomocí parametrů v~URL adrese. Těmito parametry jsou:
\begin{itemize}
    \item Povinné parametry:
    \begin{itemize}
        \item \texttt{currency\_from} -- Měna z,
        \item \texttt{currency\_to} -- Měna do,
        \item \texttt{lat} -- Zeměpisná šířka,
        \item \texttt{lng} -- Zeměpisná délka,
        \item \texttt{radius} -- Poloměr pro vyhledávání.
    \end{itemize}
    \item Nepovinné parametry:
    \begin{itemize}
        \item \texttt{amount\_from} -- Rozsah peněz od,
        \item \texttt{amount\_to} -- Rozsah peněz do,
        \item \texttt{page} -- Stránka.
    \end{itemize}
\end{itemize}

\subsection{POST /api/offer}
Zdroj pro vytvoření nové nabídky. V~těle požadavku je potřeba uvést tyto parametry:
\begin{itemize}
    \item \texttt{lat} -- Zeměpisná šířka,
    \item \texttt{lng} -- Zeměpisná délka,
    \item \texttt{radius} -- Poloměr pro vyhledávání,
    \item \texttt{amount} -- Rozsah peněz od,
    \item \texttt{comment} -- Komentář k~nabídce,
    \item \texttt{currency\_from} -- Měna z,
    \item \texttt{currency\_to} -- Měna do,
    \item \texttt{address} -- Adresa.
\end{itemize}

\subsection{GET /api/offer/\{id\}}
Zdroj pro získání informací o~jedné konkrétní nabídce.

\subsection{PATCH /api/offer/\{id\}/}
Tento zdroj slouží k~částečné úpravě nabídky. Aplikace konkrétně umožňuje pouze změnu stavu nabídky. V~těle požadavku musí být uveden parametr \texttt{status} (jeho hodnota musí rovna atributu \texttt{id} z~modelové třídy \texttt{OfferStatus}). Změny statusů musí odpovídat akcím z~diagramu \ref{fig:implementation:state-diagram} na straně \pageref{fig:implementation:state-diagram}. Příklad požadavku je uveden v~ukázce kódu \ref{code:api-status-change}.

\begin{listing}[htbp]
\caption{\label{code:api-status-change}Ukázka změny statusu přes API}
\begin{minted}[frame=lines,bgcolor=codebg,fontsize=\footnotesize,linenos,breaklines]{PYTHON}
PATCH /api/offer/123/

{
    "status":2
}
\end{minted}
\end{listing}


\subsection{GET /api/offer/\{id\}/feedback}
Slouží k~získávání dat o~hodnoceních, které se týkají dané nabídky. Výstup nelze řadit, filtrovat ani stránkovat.

\subsection{POST /api/offer/\{id\}/feedback}
Slouží k~přidání nového hodnocení k~dané nabídce. Vstupní parametry lze vypozorovat z~definice modelové třídy Feedback.

\subsection{GET /api/user}
Výpis všech uživatelů aplikace. Výpis je stránkován po 10 záznamech. Stránkování lze měnit pomocí parametru \texttt{page}.

\subsection{GET /api/user/\{id\}}
Výpis konkrétního uživatele dle zadaného parametru \texttt{id}.

\subsection{GET /api/user/\{id\}/finished\_offers}
Seznam dokončených nabídek daného uživatele.

\subsection{GET /api/user/\{id\}/user\_reaction}
Seznam nabídek čekajících na reakci daného uživatele.

\subsection{GET /api/user/\{id\}/other\_user\_reaction}
Seznam nabídek čekajících na reakci jiných uživatelů.

\subsection{GET /api/user/login/facebook}
Tento zdroj slouží pro oznámení o~registraci, resp. přihlášení, přes Facebook pomocí mobilní aplikace. Je potřeba uvést v~url parametr \texttt{access-token}, pomocí kterého si aplikace zjistí další potřebné informace z~API Facebooku.

\subsection{GET /api/language}
Výpis všech jazyků. Pokud je počet záznamů větší než 10, pak je výpis stránkován po 10 záznamech.

\subsection{GET /api/language/\{id\}}
Výpis konkrétního jazyka dle zadaného parametru \texttt{id}.

\subsection{GET /api/currency}
Výpis všech měn. Pokud je počet záznamů větší než 10, pak je výpis stránkován po 10 záznamech.

\subsection{GET /api/currency/\{id\}}
Výpis konkrétní měny dle zadaného parametru \texttt{id}.

\subsection{GET /api/status}
Výpis všech statusů nabídek.

\subsection{GET /api/status/\{id\}}
Výpis konkrétního statusu nabídky dle zadaného parametru \texttt{id}.