\begin{conclusion}
Cílem práce bylo vytvořit a otestovat webovou aplikaci v~jazyce Python, která umožní lidem zadávat nabídky, resp. poptávky, na výměnu určitého obnosu peněz z~jedné měny do druhé, fyzicky v~okolí jejich výskytu a na setkání se s~případným protějškem domluvit.
\\\\
Na základě analýzy podobných webových služeb z~kapitoly \ref{analyza} a požadavků na aplikaci z~kapitoly \ref{requirements} bylo navrženo uživatelské rozhraní. Návrhu uživatelského rozhraní se věnuje kapitola \ref{nur}. Funkčnost uživatelského rozhraní doplňuje implementace v~jazyce Python, ke kterému byl, na doporučení vedoucího práce, vybrán framework Django. Nejzajímavější části implementace jsou popsány v~kapitole \ref{implementation}. Na závěr byla aplikace podrobena testování s~uživateli. Aplikace byla také nasazena do cloudu pomocí služby heroku.com \cite{heroku} a nyní je dostupná na adrese \url{https://mhdip.herokuapp.com}. Zdrojové kódy aplikace jsou volně dostupné jak na přiloženém médiu, tak na serveru \url{www.github.com}\footnote{https://github.com/marekhanacek/diploma-thesis-app}.
\\\\
Cíl práce byl úspěšně naplněn. Aplikace splňuje jak zadání, tak všechny funkční i nefunkční požadavky, které byly na aplikaci kladeny. Výsledná aplikace umožňuje uživatelům nalézt nabídky s~tím nejlepším možným směnným kurzem. Uživatelé mají možnost nabídky prohlížet jak v~textové formě, tak ve formě mapy. Web je responzivní a tak je možné aplikaci používat i v~mobilním zařízení. Díky webovému API je aplikace připravena na rozšíření o~případnou mobilní aplikaci.
\end{conclusion}