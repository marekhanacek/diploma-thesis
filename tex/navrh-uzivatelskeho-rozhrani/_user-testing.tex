\section{Testování uživatelského rozhraní s~uživateli}
\label{user-testing}
Pro testování s~uživateli jsem připravil 10 úkolů, u~kterých očekávám, že budou ty nejčastější úkony na stránce. Testování jsem prováděl s~pěti osobami.
\subsection{Úkoly}

\begin{enumerate}
    \item Přihlaste se do systému.
    \item Změňte jazyk na češtinu.
    \item Zobrazte kontaktní informace.
    \item Zobrazte profil uživatele Jim Beam.
    \item Napište zprávu uživateli Jim Beam.
    \item Změňte adresu na vaši aktuální.
    \item Máte k~dispozici 10 000 CZK a chcete je směnit na EURa. Nalezněte nejlepší nabídku a potvrďte zájem o~tuto nabídku.
    \item Žádná z~nabídek se vám nelibí, vytvořte ji.
    \item Představte si situaci, že jste vyměnili s~jiným uživatelem peníze. Nabídka je tedy ve stavu Ready to exchange. Zaznamenejte tuto aktivitu do systému.
    \item Přidejte hodnocení ke směně v~předchozím kroku. Nabídka je nyní ve stavu Finished.
\end{enumerate}

\subsection{Představení testovaných osob}

\begin{itemize}
    \item Žena A~-- 22 let, pracující, směnárnu použila jednou v~životě, zkušený uživatel v~oblasti webu a sociálních sítí
    \item Muž B -- 53 let, pracující, méně zkušený uživatel v~oblasti webu a sociálních sítí, směnárnu užívá 1x za rok
    \item Žena C -- 49 let, pracující, méně zkušený uživatel v~oblasti webu a sociálních sítí, směnárnu nikdy nepoužila
    \item Muž D -- 23 let, student FEL ČVUT, směnárnu používá několikrát za rok, zkušený uživatel v~oblasti webu, hodně cestuje
    \item Muž E -- 28 let, pracující, směnárnu používá několikrát do roka, hodně cestuje, zkušený uživatel v~oblasti webu
\end{itemize}

\section{Shrnutí testování}
Výsledky testování a nalezené problémy můžete najít tabulkách \ref{tab:test-person-A}, \ref{tab:test-person-B}, \ref{tab:test-person-C}, \ref{tab:test-person-D} a \ref{tab:test-person-E}. Z~testování vyplývají tyto změny v~aplikaci:

\begin{itemize}
    \item Zvážit přidání vyhledávacího pole do hlavičky
    \item Tlačítko pro potvrzení akce -- přidat ikonu potvrzení
    \item Na stránce Messages text Type text here … pouze jako placeholder
    \item Přidat tlačítko pro změnu adresy i do podstránek
    \item Detail nabídky
    \begin{itemize}
        \item Designově “znehodnotit” sekci History with you
        \item Zvýraznit pole s~informacemi o~nabídce
        \item Jiné barvy pro různá tlačítka
    \end{itemize}
\end{itemize}

Všechny výše uvedené změny jsem aplikoval do aplikace.


\begin{table}[h]
    \caption{Problémy nalezené při testování osoby A}\label{tab:test-person-A}
    \begin{tabulary}{1.0\textwidth}{|L|L|L|L|}
        \hline
        \textbf{Číslo úkolu} & \textbf{Popis problému} & \textbf{Závažnost problému} \\ \hline\hline
        5 & Nevšimla si, že musí smazat původní text & 5 \\ \hline
        6 & Jako jediný z~testovaných měnila adresu přes stránku Change preferences & 5 \\ \hline
        10 & Dle uživatele je detail nabídky nepřehledný. & 5 \\ \hline
    \end{tabulary}
\end{table}

\begin{table}[h]
    \caption{Problémy nalezené při testování osoby B}\label{tab:test-person-B}
    \begin{tabulary}{1.0\textwidth}{|L|L|L|L|}
        \hline
        \textbf{Číslo úkolu} & \textbf{Popis problému} & \textbf{Závažnost problému} \\ \hline\hline
        4 & Uživatel hledal vyhledávací pole, kam může zadat hledaný výraz. Později si všiml, že se na úvodní stránce nachází odkaz na Jimův profil. & 8 \\ \hline
        5 & Na stránce se zprávami déle hledal pole pro zadání zprávy a nevšiml si, že musí smazat původní text & 5 \\ \hline
        7 & Při vyhledávání nabídek si myslel, že tlačítko New offer slouží pro potvrzení vyhledávání & 8 \\ \hline
        9 & Velmi dlouho hledal tlačítko pro potvrzení transakce. Hodně pozornosti věnoval sekci History with you. & 7 \\ \hline
    \end{tabulary}
\end{table}

\begin{table}[h]
    \caption{Problémy nalezené při testování osoby C}\label{tab:test-person-C}
    \begin{tabulary}{1.0\textwidth}{|L|L|L|L|}
        \hline
        \textbf{Číslo úkolu} & \textbf{Popis problému} & \textbf{Závažnost problému} \\ \hline\hline
        2 & Uživatel klikl na ikonu Google Translatoru v~Google Chrome pro přeložení & 0 \\ \hline
        3 & Úkol trval nad očekávání velmi dlouho & 1 \\ \hline
        4 & Uživatel hledal vyhledávací pole, kam může zadat hledaný výraz. Později si všiml, že se na úvodní stránce nachází odkaz na Jimův profil. & 8 \\ \hline
        6 & Uživatel nevěděl kde má změnit svou adresu. Později tlačítko našel. & 6 \\ \hline
        7 & Vyhledání nejlepší nabídky proběhlo v~pořádku ale, nemohla najít tlačítko pro potvrzení nabídek, což je způsobeno neznalostí anglického jazyka & 8 \\ \hline
        9 & Opět velmi dlouho hledala tlačítko pro potvrzení transakce. Pozornost brala sekce History with you. & 7 \\ \hline
    \end{tabulary}
\end{table}

\begin{table}[h]
    \caption{Problémy nalezené při testování osoby D}\label{tab:test-person-D}
    \begin{tabulary}{1.0\textwidth}{|L|L|L|L|}
        \hline
        \textbf{Číslo úkolu} & \textbf{Popis problému} & \textbf{Závažnost problému} \\ \hline\hline
        5 & Nevšiml si, že musí smazat původní text & 5 \\ \hline
        10 & Jako jediný použil možnost nejprve rozkliknout nabídku, až poté zadal své hodnocení. & 0 \\ \hline
    \end{tabulary}
\end{table}

\begin{table}[h]
    \caption{Problémy nalezené při testování osoby E}\label{tab:test-person-E}
    \begin{tabulary}{1.0\textwidth}{|L|L|L|L|}
        \hline
        \textbf{Číslo úkolu} & \textbf{Popis problému} & \textbf{Závažnost problému} \\ \hline\hline
        6 & Možnost změny adresy hledal na stránce z~předchozího úkolu -- Messages. & 8 \\ \hline
        7 & Poukázal na to, že by v~tlačítku pro potvrzení měla být ikona & 8 \\ \hline
    \end{tabulary}
\end{table}
