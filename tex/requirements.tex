\chapter{Požadavky na aplikaci}
\label{requirements}

Mnou vyvíjená webová aplikace lidem umožní zadávat nabídky/poptávky na výměnu určitého obnosu peněz z jedné měny do druhé, fyzicky v okolí jejich výskytu a na setkání se s případným protějškem domluvit.
\\\\
Následující dvě podkapitoly popisují požadavky na aplikaci z pohledu funkčních\footnote{Požadavky kladoucí omezení na funkčnost a logiku fungování systému.} a nefunkčních\footnote{Požadavky, které kladou omezení na design a provedení.} požadavků.
\\\\
Požadavky vychází jak ze zadání, tak z analýzy podobných webů.
\section{Funkční požadavky}
\begin{itemize}
    \item[F1] Nabídka pro směnu peněz obsahuje
        \begin{itemize}
            \item Měny, které budou směněny
            \item Částky v jednotlivých měnách
            \item Adresu a rádius pro vyhledávání
            \item Komentář k nabídce
            \item Hodnocení uživatele za dokončenou nabídku
        \end{itemize}
    \item[F2] Hodnocení nabídky/uživatele obsahuje
        \begin{itemize}
            \item Hodnocení pomocí 0 až 5 hvězdiček
            \item Textový komentář
            \item Hodnocení protistrany je vidno až po týdnu od předání peněz nebo ihned po zadání vlastního hodnocení
            \item Ostatní uživatelé vidí hodnocení až po zadaní hodnocení obou stran, případně po uplynutí jednoho týdne
        \end{itemize}
    \item[F3] Vyhledávání nabídky
        \begin{itemize}
            \item Vyhledávat nabídky může přihlášený i nepřihlášený uživatel
            \item Obsahuje měny a interval obnosu měn které budou směněny
            \item Lokace nebude zadávána přímo, ale bude brána ta, která je předvyplněna v profilu uživatele \footnote{U nepřihlášeného uživatele se samozřejmě nejedná o profil uživatele, ale data budou ukládána do session.}
        \end{itemize}
    \item[F4] Zadání nabídky obsahuje
        \begin{itemize}
            \item Měny a částky které budou směněny
            \item Uživatel nezadává částku do. Ta bude počítána dle aktuálního středního kurzu
            \item Lokaci, která bude předvyplněna dle adresy z profilu uživatele
        \end{itemize}
    \item[F5] Přihlášení uživatele
        \begin{itemize}
            \item Uživatelé se budou přihlašovat pouze přes Facebook API
            \item Registrace není možná
        \end{itemize}
    \item[F6] Profil uživatele obsahuje
        \begin{itemize}
            \item Jméno uživatele
            \item Čas posledního přihlášení uživatele
            \item Hodnocení uživatele
            \item Profilovou fotku ze sociální sítě Facebook
            \item Adresu používanou pro směnu peněz
            \item Rádius pro zadávání a vyhledávání nabídek
        \end{itemize}
    \item[F7] Nepřihlášený uživatel je oprávněn
        \begin{itemize}
            \item Zobrazit seznam/mapu nabídek
            \item Filtrovat nabídky dle lokality
            \item Filtrovat nabídky dle zvolených měn a částek
            \item Řadit nabídky v podobě seznamu dle definovaných kritérii
            \item Prohlížet informativní stránky webu
        \end{itemize}
\end{itemize}


\section{Nefunkční požadavky}

\begin{itemize}
    \item[N1] Dostupnost přes webové rozhraní
        \begin{itemize}
            \item Web bude přístupný na internetu pomocí webového prohlížeče
        \end{itemize}
    \item[N2] Dostupnost přes webové API
        \begin{itemize}
            \item Bude přístupné API rozhraní pro plánovanou mobilní aplikaci
        \end{itemize}
    \item[N3] Responzivní design
        \begin{itemize}
            \item Web bude optimalizován pro mobilní zařízení s minimální šířkou displeje 360 px. Pro dosažení větší přehlednosti lze pro displeje s menším rozlišením skrýt prvky, které nejsou nezbytně nutné
        \end{itemize}
    \item[N4] Optimalizace v prohlížeči
        \begin{itemize}
            \item Web bude optimalizován pro následující webové prohlížeče:
                \begin{itemize}
                    \item Google Chrome
                    \item Internet Explorer od verze 8.0
                    \item Mozilla Firefox
                    \item Android mobile
                    \item Safari
                \end{itemize}
        \end{itemize}
\end{itemize}