\department{Katedra softwarového inženýrství}
\title{Webová aplikace pro směnu peněz mezi lidmi}
\authorGN{Marek} %(křestní) jméno (jména) autora
\authorFN{Hanáček} %příjmení autora
\authorWithDegrees{Bc. Marek Hanáček} %jméno autora včetně současných akademických titulů
\supervisor{Ing. Miroslav Hrončok}
\acknowledgements{Doplňte, máte-li komu a za co děkovat. V~opačném případě úplně odstraňte tento příkaz.}
\placeForDeclarationOfAuthenticity{V~Praze}
\declarationOfAuthenticityOption{4} %volba Prohlášení (číslo 1-6)
\keywordsCS{Python, Django, směna peněz, webová aplikace}
\keywordsEN{Python, Django, currency exchange, web application}
%\website{http://site.example/thesis} %volitelná URL práce, objeví se v tiráži -- úplně odstraňte, nemáte-li URL práce


\abstractCS{Tato práce se věnuje analýze, návrhu a implementaci webové aplikace a webového API, která lidem umožní zadávat nabídky/poptávky na osobní směnu určitého obnosu peněz z~jedné měny do druhé. Uživatelé jsou tak schopni nabízet výhodnější kurzy měn než ty, jaké jsou nabízeny ve směnárnách. Na základě analýzy podobných webových služeb jsou definovány požadavky na aplikaci. Práce se dále věnuje návrhu uživatelského rozhraní a návrhu aplikace z~implementačního hlediska. Aplikace byla implementována ve frameworku Django programovacího jazyka Python a následně byla otestována s~uživateli.}
%\abstractCS{V~několika větách shrňte obsah a přínos této práce v~češtině. Po přečtení abstraktu byse čtenář měl mít čtenář dost informací pro rozhodnutí, zda chce Vaši práci číst.}
\abstractEN{This thesis is dedicated to the analysis, design and implementation of a web application and a web API that allows people to bid on a personal exchange of a certain amount of money from one currency to another. Users are thus able to offer more advantageous exchange rates than those offered in exchange office. Based on an analysis of similar web services, application requirements are defined. The thesis also deals with user interface design and application design from the implementation point of view. The application was implemented in the Django framework in the Python programming language and was then tested with users.}